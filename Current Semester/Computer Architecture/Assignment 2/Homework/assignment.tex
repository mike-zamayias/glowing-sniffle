\documentclass[a4paper,12pt]{article}
\usepackage{inconsolata}
%\renewcommand*\familydefault{\ttdefault} %% Only if the base font of the document is to be typewriter style
\usepackage[LGR, T1]{fontenc}
\usepackage{listings}
\usepackage[margin=2cm]{geometry}
\usepackage[utf8]{inputenc}
\usepackage[greek, english]{babel}
\usepackage{alphabeta}
\usepackage{amsfonts, amsmath, amssymb}
\usepackage{fixltx2e}
\usepackage{subfig}
\usepackage{float}
\usepackage{xcolor}
\usepackage{type1cm, type1ec}


\definecolor{codeblue}{RGB}{44,133,217}
\definecolor{codegray}{RGB}{138,150,150}
\definecolor{codepurple}{RGB}{0.58,0,0.82}
\definecolor{backcolour}{RGB}{220,220,220}
\lstdefinestyle{mystyle}{
    backgroundcolor=\color{backcolour},
    commentstyle=\color{codegray},
    keywordstyle=\color{codeblue},
    numberstyle=\color{codegray}\ttfamily\bfseries,
    stringstyle=\color{codepurple},
    basicstyle=\ttfamily\small\bfseries,
    breakatwhitespace=false,
    breaklines=true,
    captionpos=b,
    keepspaces=true,
    numbers=left,
    numbersep=5pt,
    showspaces=true,
    showstringspaces=false,
    showtabs=false,
    tabsize=2
}
\lstset{style=mystyle}


\begin{document}

\begin{titlepage}
    \begin{center}
        \vspace*{\fill}
        \huge{\textbf{Αρχιτεκτονική Υπολογιστών}}
        \vfill
        Εργαστήριο 2
        \vfill
        \vspace*{\fill}
        \vfill
        \normalsize\textbf{Ζαμάγιας Μιχάλης\\tp5000\\}
        \small\textbf{\today}
        \vfill
    \end{center}
\end{titlepage}



\section*{Άσκηση 1}
Δίνεται ο παρακάτω κώδικας:
\begin{lstlisting}
.text
.globl main
main:
    addi $16, $0, 20
    addi $17, $0, 10
    beq $16, $17, label1
    sub $18, $17, $16
    j label2
label1:
    sub $18, $16, $17
label2:
    j main
\end{lstlisting}

\begin{enumerate}
    \item \begin{itemize}
              \item Εκτελέστε το πρόγραμμα και γράψτε την τιμή του
                    καταχωρητή \textbf{\$18} (σε δεκαεξαδικό).\\
                    \textbf{Απάντηση:} \texttt{fffffff6}
              \item Ποιες γραμμές κώδικα εκτελέστηκαν;\\
                    \textbf{Απάντηση:} Eκτελέστηκαν οι γραμμές 4, 5, 6, 7, 8 και 12.
          \end{itemize}
    \item \begin{itemize}
              \item Γράψτε ποια θα είναι η νέα τιμή του καταχωρητή \textbf{\$18} (σε δεκαεξαδικό) αν ο παραπάνω κωδικάς αλλάξει ως εξής:
                    \begin{lstlisting}[firstnumber=4]
addi $16, $0, 10
addi $17, $0, 10
\end{lstlisting}
                    \textbf{Απάντηση:} \texttt{0}
              \item Ποιες γραμμές κώδικα εκτελέστηκαν;\\
                    \textbf{Απάντηση:} Eκτελέστηκαν οι γραμμές 4, 5, 6, 10 και 12.
          \end{itemize}
\end{enumerate}

\newpage
\section*{Άσκηση 2}
Γράψτε σε κώδικα που εκτελεί την παρακάτω λειτουργία:
\begin{lstlisting}
x = 0
for i = 1 to 5
    x = x + i
\end{lstlisting}
\textbf{Απάντηση:}
\begin{lstlisting}
.text
.globl main
main:
    #   $16 = x, $17 = i, $18 = c
    addi	$16, $0, 0			# $16 = $0 + 0
    addi	$17, $17, 1			# $17 = $17 + 1
    addi	$18, $0, 6			# $18 = $0 + 6
    beq		$0, $0, target  	# if $0 == $0 then target
target:
    add		$16, $16, $17		# $16 = $16 + $17
    addi	$17, $17, 1			# $17 = $17 + 1
    bne		$17, $18, target	# if $17 != $18 then target
    j		main				# jump to main
\end{lstlisting}

\end{document}