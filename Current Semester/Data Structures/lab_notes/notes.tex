\documentclass[14pt, fleqn, leqno]{extreport}
\usepackage{perpage}
%\documentclass[12pt]{extreport}
\usepackage[margin=2cm,includeheadfoot,a4paper]{geometry}
\usepackage{fontspec}
%\usepackage[utf8x]{inputenc}
\usepackage[english,greek]{babel}
\usepackage{indentfirst}
\usepackage[dvipsnames]{xcolor}
\usepackage{listings}
\usepackage{titlesec}
\usepackage{xifthen, xparse}
\usepackage{fancyhdr}
%\usepackage{fancyvrb}
\usepackage[autostyle,english=american]{csquotes}
\usepackage[hyphens]{url}
\usepackage{hyperref}

\MakePerPage{footnote} 

%\setlength{\mathindent}{0pt}
\setlength{\headheight}{17pt}

%\renewcommand{\arraystretch}{1.5}

\titleformat{\chapter}[display]
  {\normalfont\bfseries}{}{0pt}{\Huge}

\setlength{\parskip}{0cm}
\setlength{\parindent}{1cm}

\setmainfont{[EBGaramond-Regular.ttf]}
\setmonofont{[FiraMono-Regular.otf]}

\MakeOuterQuote{"}

\hypersetup{
    colorlinks = true,
    linkcolor=black,
    filecolor=magenta,
    urlcolor=blue,
    pdftitle={Project 2}
}


%\definecolor{name}{model}{color-spec}

\lstdefinestyle{mystyle}{
    language=C,
    backgroundcolor=\color{white},   
    commentstyle=\color{teal},
    keywordstyle=\color{blue},
    numberstyle=\color{gray}\ttfamily,
    stringstyle=\color{orange},
    basicstyle=\ttfamily\footnotesize,
    breakatwhitespace=false,         
    breaklines=true,                 
    captionpos=b,                    
    keepspaces=true,                 
    numbers=left,                    
    numbersep=5pt,                  
    showspaces=false,                
    showstringspaces=false,
    showtabs=false,                  
    tabsize=2,
    frame=lines,
    framesep=0.1cm,
    rulecolor=\color{black},
    morestring=[b]"
}

\lstset{style=mystyle}

\pagenumbering{arabic}

\fancyhf{}
\pagestyle{fancy}
\lhead{\small \rightmark}
\cfoot{\thepage}

\begin{document}

\title{Δομές Δεδομένων\\Σημειώσεις εργαστηρίου}
\author{Μιχαήλ Ανάργυρος Ζαμάγιας\\ΤΠ5000}
\date{\today}
\maketitle
\newpage

\tableofcontents

\newpage
\chapter{Εργαστήριο 1}

\section{Μονοδιάστατος πίνακας}

Ισχύουν τα εξής:
\begin{lstlisting}
pin == &pin[0]
pin+k == &pin[k]
*pin == pin[0]
*(pin+k) == pin[k]
\end{lstlisting}

Η τιμή ενός δείκτη ισούται με τη διεύθυνση μνήμης του byte στο οποίο είναι τοποθετημένος ο δείκτης και εμφανίζεται στην οθόνη με την χρήση του προσδιοριστή \lstinline{%p}.

\section{Πίνακας ως ορίσματα συνάρτησης}

Για να "περάσω" σε μια συνάρτηση ως παράμετρο ένα πίνακα, περνάω ένα δείκτη στην αρχή του πίνακα και (αν χρειάζεται) το μέγεθος του πίνακα.

Στον ορισμό μιας συνάρτησης (για παράδειγμα, της \lstinline{parad}) οι παρακάτω συμβολισμοί είναι ισοδύναμοι:
\begin{lstlisting}
void parad(int *pin)
void parad(int pin[])
\end{lstlisting}
Σε κάθε περίπτωση, το \lstinline{pin} είναι δείκτης σε ακέραιο.

\section{Ταξινόμηση πίνακα}

Υπάρχουν διάφοροι αλγόριθμοι ταξινόμησης ενός πίνακα. Αυτός που περιγράφεται εδώ είναι γνωστός ως "Ταξινόμηση με επιλογή".

Η συνάρτηση θα ξεκινά από το στοιχείο της πρώτης θέσης τοτ πίνακα, το \lstinline{list[0]}, με στόχο να τοποθετηθεί στην θέση αυτή η μικρότερη τιμή του πίνακα.Η συνάρτηση να διατρέχει όλα τα υπόλοιπα στοιχεία, από το \lstinline{list[1]} μέχρι το \lstinline{list[N-1]} και να συγκρίνει καθένα με το πρώτο. Αν βρει κάποιο μικρότερο από το πρώτο, τα στοιχεία εναλλάσσονται μεταξύ τους. Τελικά το \lstinline{list[0]} θα έχει την μικρότερη τιμή του πίνακα.

Αφού τελειώσουμε με το πρώτο στοιχείο επαναλαμβάνεται η διαδικασία, προσπαθώντας να βάλουμε στη θέση 1 του πίνακα τη δεύτερη σε μέγεθος τιμή. Συγκρίνονται δηλαδή όλα τα στοιχεία από το \lstinline{list[2]} και μετά με το \lstinline{list[1]}. Αν βρεθεί κάποιο στροιχείο μικρότερο από το \lstinline{list[1]}, τα στοιχεία εναλλάσσονται μεταξύ τους, κ.ο.κ.

Χρειάζεστε δύο εμφωλευμένες επαναλήψεις.

\chapter{Εργαστήριο 2}

\section{Δισδιάστατος πίνακας}

Αναφερόμαστε σε κάθε στοιχείο ενός πίνακα δύο διαστάσεων χρησιμοποιώντας δύο αριθμητικές ετικέτες, για παράδειγμα το στοιχείο της γραμμής \lstinline{3}, της στήλης \lstinline{5} ενός πίνακα \lstinline{pin}, λέγεται \lstinline{pin[3][5]}.

Χρειαζόμαστε για να διαβάσουμε ή να γράψουμε τα στοιχεία ενός πίνακα δύο διαστάσεων διπλή επαναληπτική εντολή (εμφωλευμένες επαναλήψεις).

Κατά το πέρασμα ενός πίνακα δύο διαστάσεων σε μια συνάρτηση πρέπει να "περνάμε" στην συνάρτηση τον αριθμό των στηλών του πίνακα.

\section{Πίνακας συμβολοσειρών}

Με δεδομένη την δήλωση:
\begin{center}
    \lstinline{char students[R][C];}
\end{center}
\begin{itemize}
    \item Η \lstinline{k} γραμμή του πίνακα είναι μονοδιάστατος πίνακας χαρακτήρων και λέγεται\\\lstinline{students[k]}.
    \item To \lstinline{students[k]} είναι πίνακας χαρακτήρων, άρα και δείκτης σε χαρακτήρα.
    \item Η ταξινόμηση (κατάταξη αλφαβητικά) συμβολοσειρών μοιρεί να γίνει με τον ίδιο αλγόριθμο που γίνεται η ταξινόμηση ακεραίων. Η σύγκριση των συμβολοσειρών θα γίνεται με την συνάρτηση \lstinline{strcmp()}.
\end{itemize}

\chapter{Εργαστήριο 3}

\section{Δομή (struct): περιγραφή, περδία δομής, δηλώσεις και δεδομένα}

Η περιγραφή της δομής προϋπάρχει της δήλωσης για να "διδάξει" στον compiler πως είναι το νέο είδος μεταβλητών που δημιουργούμε.

Η περιγραφή της δομής βρίσκεται ανάμεσα σε άγκιστρα, μετά τα οποία υπάρχει το \lstinline{;}.

Αναφερόμαστε στο κάθε πεδίο μιας δομής ως εξής: \lstinline{struct_name.member_name}.

Η εμφάνιση μόνο του είδους της δομής (της λέξης δηλαδή που ακολυθεί την λέξη struct) αποτελεί συντακτικό λάθος.

Η εμφάνιση μόνο του ονόματος κάποιου πεδίου της δομής (των λέξεων δηλαδή που εμφανίζονται στην περιγραφή μιας δομής) αποτελεί επίσης συντακτικό λάθος.

Μια δομή \lstinline{a} κάποιου τύπου μπορεί να γίνει ίση με μια δομή \lstinline{b} του ίδιου τύπου με απλή απόδοση τιμής, δηλαδή:
\begin{center}
    \lstinline{a = b;}
\end{center}

\section{Φωλιασμένη δομή}

Φωλιασμένη δομή λέγεται η δομή των οποίων κάποιο ή κάποια πεδία είναι δομή, η οποία έχει ήδη περιγραφεί προηγουμένως.

Η προσπέλαση των πεδίων σε Φωλιασμένη δομή γίνεται με την πολλαπλή χρήση της τελείας (\lstinline{.}).

\section{Πίνακας δομών}

Σε ένα πίνακα δομών, κάθε στοιχείο του πίνακα είναι μια δομή. Στην άσκηση 3, το \lstinline{pin} είναι ένας πίνακας \lstinline{N} δομών του είδους \lstinline{student}. Άρα, το \lstinline{pin[0]}, \lstinline{pin[0]} κ.λ.π. είναι δομές του είδους \lstinline{student}.

Για παράδειγμα ο χαρακτήρας της τρίτης θέσης του πίνακα \lstinline{name} του τέταρτου στοιχείου του πίνακα \lstinline{pin} λέγεται \lstinline{pin[3].name[2]}.

Το \lstinline{fflush(stdin)} αδειάζει τον buffer εισόδου κι έτσι η \lstinline{gets} δεν επηρεάζεται από το \lstinline{Enter} της \lstinline{scanf} που έχει προηγηθεί.

\newpage
Η άσκηση 3:
\lstinputlisting{code/lab3_task3.c}

\section{Δομή ως παράμετρος και ως τιμή επιστροφής συναρτήσεων}

Μια συνάρτηση μπορεί να δέχεται ως ορίσματα δομές, όπως οποιοδήποτε άλλο τύπο δεδομένων. Για παράδειγμα η επικεφαλίδα μιας συνάρτησης που είναι \lstinline{void} και δέχεται ως παραμέτρους δύο δομές, τις \lstinline{s1} και \lstinline{s2} του τύπου \lstinline{stype}, θα είναι:
\begin{center}
    \lstinline{void example(struct stype s1, struct stype s2)}
\end{center}

Η κλήση της συνάρτησης θα είναι:
\begin{center}
    \lstinline{example(s1, s2);}
\end{center}

Μια συνάρτηση μπορεί να δέχεται ως όρσιμα πίνακα δομών, όπως και οποιοδήποτε άλλο πίνακα. Για παράδειγμα η επικεφαλίδα μια συνάρτησης που είναι \lstinline{void} και δέχεται ως παράμετρο ένα πίνακα δομών του τύπου \lstinline{funds}, τον \lstinline{pin}, θα είναι:
\begin{center}
    \lstinline{void example(struct funds *pin)}\\
    ή\\
    \lstinline{void example(struct funds pin[])}
\end{center}

Η κλήση της συνάρτησης θα είναι:
\begin{center}
    \lstinline{example(pin);}
\end{center}

Μια συνάρτηση μπορεί να έχει τιμή επιστροφής δομή, όπως και οποιοδήποτε άλλο είδος δεδομένων. Για παράδειγμα η επικεφαλίδα μιας συνάρτησης που είναι \lstinline{void} και δέχεται ως μια δομή του τύπου \lstinline{funds}, την \lstinline{str}, θα είναι:
\begin{center}
    \lstinline{struct funds reading (struct funds str)}
\end{center}

Η κλήση της συνάρτησης θα είναι:
\begin{center}
    \lstinline{x = reading(str);}
\end{center}



\chapter{Εργαστήριο 4}

\section{Δυναμική δέσμευση μνήμης (συνάρτηση \lstinline{malloc})}



\section{Πίνακες δεικτών}

\chapter{Εργαστήριο 5}

\section{Στοίβες, υλοποίηση με πίνακα}

\chapter{Εργαστήριο 6}

\section{Απλά συνδεδεμένες λίστες (δημιουργία)}
\section{Λειτουργίες στις απλά συνδεδεμένες λίστες: αναζήτηση, εισαγωγή, διαγραφή, μετακίνηση, συνένωση λιστών}



\end{document}