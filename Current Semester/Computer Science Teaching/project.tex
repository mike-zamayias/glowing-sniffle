\documentclass[a4paper,14pt]{extreport}
\usepackage[margin=2cm]{geometry}
\usepackage{fontspec}
\usepackage[utf8x]{inputenc}
\usepackage[english,greek]{babel}
\usepackage{indentfirst}
%\usepackage[none]{hyphenat}
\usepackage{fancyhdr}
\usepackage{epigraph}
\usepackage[dvipsnames]{xcolor}
\usepackage{listings}
\usepackage[hyphens]{url} 
\usepackage{hyperref}
\usepackage{titlesec}

\titleformat{\chapter}[display]
  {\normalfont\bfseries}{}{0pt}{\Huge}

\setlength{\parskip}{0cm}
\setlength{\parindent}{1cm}

\setmainfont{[EBGaramond-Regular.ttf]}
\setmonofont{[FiraMono-Regular.otf]}

\hypersetup{
    colorlinks = true,
    linkcolor=black,
    filecolor=magenta,      
    urlcolor=blue,
    pdftitle={Προγραμματισμός, σε Python}
}

\lstdefinestyle{mystyle}{
    basicstyle=\ttfamily\small,
    breakatwhitespace=false,         
    breaklines=true,                 
    captionpos=b,                    
    keepspaces=true,                 
    numbers=left,                    
    numbersep=5pt,                  
    showspaces=false,                
    showstringspaces=false,
    showtabs=false,                  
    tabsize=4
}
\lstset{style=mystyle}

\renewcommand {\epigraphflush}{center}

\pagestyle{fancy}
\fancyhf{}
\rhead{
    Προγραμματισμός, σε Python
    }
\lhead{
    Διδακτική της Πληροφορικής
}
\cfoot{\thepage}
        
\title{Προγραμματισμός, σε Python}
\author{Μιχαήλ Ανάργυρος Ζαμάγιας}
\date{\today}

\begin{document}

\maketitle

\tableofcontents

\chapter{Εισαγωγή}
\newpage
\section{Πρόλογος}
Άρχισα να προγραμματίζω, να γράφω Python συγκεκριμένα, στην πρώτη λυκείου ως χόμπι. Είχα χάσει το ενδιαφέρον που είχα για το hardware από το δημοτικό, ενώ παράλληλα κέρδιζε το ενδιαφέρον μου η διαδικασία του να κάνω το hardware χρήσιμο για τον χρήστη, σε αυτή την περίπτωση εμένα.

Στην δευτέρα λυκείου περίπου, είχα καταφέρει να γράψω ένα bot που έτρεχε στο terminal του υπολογιστή μου. Μπορούσε να απαντήσει όλες τις ερωτήσεις που του έκανα, δεδομένου, φυσικά, ότι τις καταλάβαινε. Δεν μπορούσε όμως να κρατήσει διάλογο. Άλλωστε ήταν ένα χαζό bot, όχι κάποια γενική τεχνιτή νοημοσύνη.

Ακόμα, είχα γράψει ένα πρόγραμμα "μετερεολόγο" για μία σχολική εκδρομή, της δευτέρας ή τρίτης λυκείου. Ήταν μεγαλύτερο πρόγραμμα σε σύγκριση με το bot, το  οποίο μιλούσε με ένα API για τα δεδομένα και τα εμφάνιζε σε μορφή ευνόητη στον άνθρωπο. Όμως, τα δελτία δεν ήταν ακριβή, συγκριτικά με άλλες υπηρεσίες, με αποτέλεσμα να σταματήσω να ασχολούμαι με αυτό.

Μερικά χρόνια μετά, θέλω να σας εξοικειώσω με τα βασικά της Python μέσω αυτού του εγχειριδίου. Δεν χρειάζονται ανεπτυγμένες γνώσεις προγραμματισμού για να γράψετε κάτι το οποίο θα σας διευκολύνει με τις δουλειές σας στον υπολογιστή, π.χ. την αυτοματοποίηση μιας διαδικασίας.
\section{Γιατί να ασχοληθώ με τον προγραμματισμό;}

\epigraph{"Codes are a puzzle. A game, just like any other game."}{— Alarn Turing}
Ο προγραμματισμός, ως γνώση, δίνει σε κάποιον την δεξιότητα να
μετατρέπει ένα πρόβλημα σε κάτι που μπορεί να κατανοήσει και να λύσει,
τις περισσότερες φορές, ένας υπολογιστής. Μάλιστα, θα υποστήριζα ότι
είναι μια βασική δεξιότητα για κάποιον άνθρωπο στην εποχή μας,
εάν εκείνος χρησιμοποιεί υπολογιστές.

\section{Γιατί να μάθω Python;}

Η Python είναι μια δερμηνευμένη, γενικού σκοπού και υψηλού επιπέδου γλώσσα
προγραμματισμού. Δηλαδή:
\begin{itemize}
    \item μέρη προγράμματος μπορούν να εκτελστούν εκτός του κυρίου
          προγράμματος, μέσα από τον διερμηνέα,
    \item βρίσκεται πίσω από πολλές εφαρμογές και προσφέρει λύση σε πολλά
          προβλήματα, και
    \item το συντακτικό της είναι απλούστερο, ευκολότερο στην κατανόηση σε
          σύγκριση με άλλες γλώσσες προγραμματισμού.
\end{itemize}

\section{Ποιος είναι ο σκοπός αυτού του εγχειριδίου;}

Στόχος μου μέσα από αυτό το εγχειρίδιο είναι η εξοικείωση του αναγνώστη
με την Python και η εισαγωγή του στον τρόπο σκέψης ενός προγραμματιστή.
Το εγχειρίδιο χωρίζεται σε δύο μέρη. Στο πρώτο μέρος ο αναγνώστης μαθαίνει
για τα βασικά της γλώσσας, βασικές έννοιες προγραμματισμού, είδη δεδομένων
και τους τρόπους χρήσης τους. Στο δεύτερο μέρος, ο αναγνώστης θα εφαρμόσει
πρακτικά όσα έμαθε στο πρώτο μέρος σε μικρά προγράμματα.

\chapter{Τα βασικά}

Τα βασικά είναι βαρετά, λόγω της φύσης της γλώσσας δεν χρειάζεστε να επιμείνετε στα βασικά. Είναι αρκετά εύκολο, με μια καλή κατανόηση, των βασικών εργαλείων της γλώσσας να χτίσετε αρκετά πράγματα. Τα βασικά εργαλεία σε αυτήν την περίπτωση είναι οι προτάσεις υπό συνθήκη, οι επαναλήψεις, οι συναρτήσεις και τα δομοστοιχεία της γλώσσας. Ο αναγνώστης πρέπει να ασχοληθεί περισσότερο με το πρακτικό κομμάτι, με project, όπως π.χ. την δημιουργεία εργαλείων που θα του γλιτώσει αρκετό χρόνο από συγκεκριμένες ρουτίνες στον υπολογιστή.
\newpage

\section{Ξεκινώντας}

Η Python δεν έρχεται προεγκατεστημένη με τα Windows και συνήθως στις Linux διανομές η προεγκατεστημένη έκδοση είναι η 2.7 η οποία πλέον έχει σταματήσει να λαμβάνει ενημερώσεις και υποστήριξη. Άρα, χρειάζεστε μια νεότερη έκδοση της  Python. Η Python είναι μια cross-platform γλώσσα προγραμματισμού, με υποστήριξη σε Windows, Linux και macOS και με μερικές αλλαγές ανά την πλατφόρμα. Σε αυτό το εγχειρίδιο χρησιμοποιούνται οι \lstinline{Python 3.7.6} και \lstinline{conda 4.8.3} στην διανομή Linux \lstinline{Pop!_OS}. Τέλος, τα προγράμματα γραμμένα σε Python έχουν κατάληξη \lstinline{.py}.

\subsection{Εγκατάσταση Python}

Προτείνω στον αναγνώστη να εγκαταστήσει την Anaconda, η οποία είναι μια δωρεάν και  ανοιχτού κώδικα διανομή της Python, καθώς προσφέρει ένα μεγάλο σύνολο εργαλείων σε ένα σημείο. Η διαδικασία εγκατάστασης παραμένει εύκολη και καθιστά μελλοντικούς πειραματισμούς πιο εύκολους λόγω του πλήθους εργαλείων που περιέχει.

\subsubsection{Σε Windows}

Επισκεφθείτε την \href{https://tinyurl.com/yc39u67t}{επίσημη σελίδα}, κατεβάστε το αρχείο εγκατάστασης για Windows με την τελευταία έκδοση της Python και ξεκινήστε την εγκατάσταση. Δεν χρειάζετε αλλάξετε τις προεπιλεγμένες ρυθμίσεις.

\subsubsection{Σε Linux}

Επισκεφθείτε την \href{https://tinyurl.com/yc39u67t}{επίσημη σελίδα} και κατεβάστε το αρχείο εγκατάστασης για Linux με την τελευταία έκδοση της Python. Ανοίξτε το terminal στην τοποθεσία που έγινε η λήψη του αρχείου εγκατάστασης κι εκτελέστε την εντολή \lstinline{bash ./"Anaconda"}, όπου \lstinline{"Anaconda"} το όνομα αρχείου εγκατάστασης. Η εγκατάσταση θα ξεκινήσει στο terminal, και προχωράτε την πρόοδο της εγκατάστασης πατώντας \lstinline{Enter}. Στο σημείο\\ \lstinline{Do you wish the installer to initialize Anaconda3 by running}\\ \lstinline{conda init?}

απαντήστε \lstinline{yes}, έτσι ώστε να έχετε πιο εύκολη πρόσβαση στην γλώσσα την επόμενη φορα που θα χρησιμοποιείσετε την γλώσσα.

\subsubsection{Επιβεβαίωση εγκατάστασης}

Σε Windows, αναζητήστε το \lstinline{Anaconda Prompt}
στο μενού έναρξης ή σε Linux, ανοίξτε το terminal.
Εάν η εγκατάσταση ολοκληρώθηκε με επιτυχία, η εκτέλεση των δυο επόμενων
εντολών, \lstinline{python --version} και
\lstinline{conda --version}, θα έχει ένα παρόμοιο
αποτέλεσμα: \lstinline{Python 3.7.6} και
\lstinline{conda 4.8.3}, αντίστοιχα.

\subsection{Εγκατάσταση επεξεργαστή κειμένου}

Μπορούμε να γράψουμε Python σε οποιονδήποτε επεξεργαστή κειμένου, ακόμα και στο Σημειωματάριο που έρχεται με τον υπολογιστή μας. Υπάρχουν όμως άλλα προγράμματα τα οποία είναι φτιαγμένα για προγραμματισμό, τα οποία φέρουν χαρακτηριστικά που θα κάνουν την ζωή μας πιο εύκολη στην συνέχεια. Μερικά τέτοια δωρεάν και ανοιχτού κώδικα προγράμματα είναι τα εξής: \href{https://tinyurl.com/hh3vvrn}{Sublime}, \href{https://tinyurl.com/l2uxckn}{Atom} και \href{https://tinyurl.com/y54gclet}{VSCodium}. Παρακάτω θα δείτε πως να εγκαταστήσετε το VSCodium.

\subsubsection{Σε Windows}

Επισκεφθείτε την \href{https://tinyurl.com/ybqxegq3}{επίσημη σελίδα} και πατήστε\\\lstinline{Download latest release here}

Κατεβάστε το αρχείο εγκατάστασης\\\lstinline{VSCodiumSetup-x64-1.45.1.exe}

και ξεκινήστε την εγκατάσταση. Δεν χρειάζετε να αλλάξετε τις προεπιλεγμένες ρυθμίσεις.

\subsubsection{Σε Linux}
Επισκεφθείτε την \href{https://tinyurl.com/ybqxegq3}{επίσημη σελίδα} και
ακολουθήστε τις οδηγίες εγκατάστασης για την διανομή σας.
\subsubsection{Επιβεβαίωση εγκατάστασης}
Σε οποιοδήποτε λειτουργικό σύστημα, αναζητήστε για "VSCodium" στην λίστα
προγραμμάτων σας. Εάν μπορείτε να το εκτελέσετε από εκεί, το έχετε εγκαταστήσει
επιτυχώς. Διαφορετικά ξεκινήστε την διαδικασία εγκατάστασης από την αρχή.
\section{Η γλώσσα}
\subsection{Διερμηνευτής}
\paragraph{
    Ο διερμηνευτής της Python επιτρέπει στον χρήστη να εκτελέσει διαδραστικά
    εντολές Python, κάτι που μπορεί να φανεί χρήσιμο για πειραματισμό ή
    την δοκιμή κομματιών προγράμματος. Μπορείτε να χρησιμοποιήσετε τον
    διερμηνευτή,
}
\begin{itemize}
    \item ανοίγοντας το Anaconda Prompt στα Windows και εκτελώντας την εντολή
          \lstinline{python}.
    \item ανοίγοντας το terminal στα Linux και εκτελώντας την εντολή
          \lstinline{python}.
\end{itemize}
Αυτός είναι ο διερμηνευτής:
\begin{lstlisting}[language=Python]
Python 3.7.6 (default, Jan  8 2020, 19:59:22)
[GCC 7.3.0] :: Anaconda, Inc. on linux
Type "help", "copyright", "credits" or "license" for more information.
>>>
\end{lstlisting}
Για να εκτυπώσετε την φράση "hello world" στον διερμηνευτή εκτελείτε  την
εντολή \\
\lstinline{print("hello world")}:
\begin{lstlisting}[language=Python]
>>> print("hello world")
hello world
\end{lstlisting}
Για να φύγετε από τον διερμηνευτή, αρκεί να εκτελέσετε
\lstinline{quit()}.
Ακόμα, μέσω διερμηνευτή εκτελείτε τα προγράμματα σας.Ας υποθέσουμε ότι
έχουμε το πρόγραμμα \lstinline{hello_world.py},
το οποίο περιέχει την εντολή \lstinline{print("hello world")}.
Για να εκτελέσουμε το πρόγραμμά μας, αρκεί να εκτελέσουμε \\
\lstinline{python hello_world.py}
στο terminal:
\begin{lstlisting}[language=Python]
python hello_world.py
hello world    
\end{lstlisting}
\subsection{Εκτέλεση προγραμμάτων}
\subsection{Εύρεση κι επίλυση σφαλμάτων}
\subsection{Σχόλια}
\subsection{Μεταβλητές}
\subsection{Τύποι δεδομένων}
\subsection{Συμβολοσειρές}
\subsubsection{Μορφοποίηση}
\subsubsection{Μέθοδοι}
\subsection{Λίστες}
\subsection{Tuples}
\subsection{Σύνολα}
\subsection{Λεξικά}
\subsection{Προτάσεις υπό συνθήκη}
\subsection{Βρόγχοι}
\subsection{Δομοστοιχεία}
\subsection{Δουλεύοντας με αρχεία}
\subsection{Δουλεύοντας με JSON αρχεία}

\chapter{Προγράμματα}

\chapter{Βιβλιογραφία}
\epigraph{\href{https://tinyurl.com/ycnad9ch}
    {Total 0 knowledge, entirely parallel with programming of any kind.
        Heard Python is simple, why would I want to learn it ?
    }
}{u/Azsras\_Zuralix on r/learnpython}
\section{Βιβλία}
\begin{itemize}
    \item \href{https://tinyurl.com/y7l2a48c}{Python Crash Course, 2nd
              Edition — by Eric Matthes}
\end{itemize}
\section{Video}
\begin{itemize}
    \item \href{https://tinyurl.com/ya8wk4xm}{Python Crash Course}
\end{itemize}
\section{Σύνδεσμοι}
\begin{itemize}
    \item \href{https://tinyurl.com/yyzfa2bg}{WordReference Dictionary}
    \item\href{https://tinyurl.com/o5vxal7}{Λεξικό της κοινής νεοελληνικής}
    \item \href{https://tinyurl.com/y9q2elk4}{Βιβλιογραφία, Wikipedia}
    \item \href{https://tinyurl.com/y9g9nkh2}{Python, Wikipedia}
    \item \href{https://tinyurl.com/ycy6jsw5}{Anaconda (Python distribution),
              Wikipedia}
    \item \href{https://tinyurl.com/y7rogsec}{Anaconda Individiual Edition,
              Anaconda | The World's Most Popular Data Science Platform}
    \item \href{https://tinyurl.com/ogoqf2p}{Conditional statements, Wikipedia}
    \item \href{https://tinyurl.com/y8y59y44}{Python Cheatsheet}
    \item \href{https://tinyurl.com/y54gclet}{VSCodium is a community-driven,
              freely-licensed binary distribution of Microsoft’s editor VSCode}
\end{itemize}
\section{Χρήσιμα αρχεία}
\begin{itemize}
    \item \href{https://tinyurl.com/y9l8o5n6}{Βιβλιογραφική ανασκόπηση,
              Δημοκρίτειο Πανεπιστήμιο Θράκης}
    \item \href{https://tinyurl.com/yaaswz5p}{Εισαγωγή στη LaTeX για φοιτητές.
              (An Introduction to Latex in Greek)}
    \item \href{https://tinyurl.com/nqbrvss}{Python Cheat Sheet}
\end{itemize}

\end{document}