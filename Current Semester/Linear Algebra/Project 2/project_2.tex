\documentclass[12pt]{extreport}
\usepackage[margin=2cm,includeheadfoot,a4paper]{geometry}
\usepackage{fontspec}
%\usepackage[utf8x]{inputenc}
\usepackage[english,greek]{babel}
\usepackage{indentfirst}
\usepackage[dvipsnames]{xcolor}
\usepackage{listings}
\usepackage{titlesec}
\usepackage{amsmath, mathtools}
\usepackage{xifthen, xparse}
\usepackage{fancyhdr}
\usepackage{fancyvrb}
\usepackage[hyphens]{url}
\usepackage{hyperref}

\setlength{\headheight}{17pt}

%\renewcommand{\arraystretch}{1.5}

\titleformat{\chapter}[display]
  {\normalfont\bfseries}{}{0pt}{\Huge}

\setlength{\parskip}{0cm}
\setlength{\parindent}{1cm}

\setmainfont{[EBGaramond-Regular.ttf]}
\setmonofont{[FiraMono-Regular.otf]}

\hypersetup{
    colorlinks = true,
    linkcolor=black,
    filecolor=magenta,
    urlcolor=blue,
    pdftitle={Project 2}
}


%\definecolor{name}{model}{color-spec}

\lstdefinestyle{mystyle}{
    language=Octave,
    backgroundcolor=\color{white},   
    commentstyle=\color{teal},
    keywordstyle=\color{blue},
    numberstyle=\color{gray}\ttfamily,
    stringstyle=\color{orange},
    basicstyle=\ttfamily\footnotesize,
    breakatwhitespace=false,         
    breaklines=true,                 
    captionpos=b,                    
    keepspaces=true,                 
    numbers=left,                    
    numbersep=5pt,                  
    showspaces=false,                
    showstringspaces=false,
    showtabs=false,                  
    tabsize=2,
    frame=lines,
    framesep=0.1cm,
    rulecolor=\color{black},
    morestring=[b]"
}

\lstset{style=mystyle}

\pagenumbering{arabic}

\pagestyle{fancy}
\fancyhf{}
\rhead{Project 2}
\lhead{Αριθμητική Γραμμική Άλγεβρα}
\cfoot{\thepage}

\newcommand\rowop[1]{\scriptstyle\smash{\xrightarrow[\vphantom{#1}]{\mkern-4mu#1\mkern-4mu}}}

\DeclareDocumentCommand\converttorows
{>{\SplitList{,}}m}
{\ProcessList{#1}{\converttorow}}
\NewDocumentCommand{\converttorow}{m}
{\ifthenelse{\isempty{#1}}{}{\rowop{#1}}\\}

\DeclareDocumentCommand \rowops{m}
{\;
 \begin{matrix}
\converttorows {#1}
 \end{matrix}
 \; }


\begin{document}

\title{Αριθμητική Γραμμική Άλγεβρα\\Project 2}
\author{Μιχαήλ Ανάργυρος Ζαμάγιας\\TP5000}
\date{\today}
\maketitle
\newpage

\tableofcontents

\chapter{Άσκηση 1}
\section{Εκφώνηση}

Δίνεται το σύστημα \eqref{eq:11}. Γράψτε αυτό το σύστημα στην μορφή $A\bar{x}=\bar{b}$ για κατάλληλα διανύσματα $\bar{x}$ και $\bar{b}$. Εδώ $\mu$ είναι το τελευταίο ψηφίο του αριθμού μητρώου σας. Φτιάξτε ένα πρόγραμμα που να δέχεται ως δεδομένα έναν $3x3$ πίνακα $A$ και ένα $3x1$ διάνυσμα $\bar{b}$ και να είναι σε θεση να μας πει αν το σύστημα είναι αόριστο ή αδύνατο ή έχει μοναδική λύση. Αν το σύστημα έχει μοναδική λύση πρέπει να δίνει αυτή την λύση υπολογισμένη με την μέθοδο του επαυξημένου πίνακα (απαλοιφή Gauss). Δώστε τα αποτελέσματα του προγράμματός σας στο σύστημα εξισώσεων \eqref{eq:11} καθώς και στο σύστημα \eqref{eq:12}.

\begin{equation}%eq11
    \begin{aligned}
        3x      +   4y    + 2z & = 3 \\
        x       -   2y         & = 3 \\
        \mu x   +   3y    + z  & = 2
    \end{aligned}\label{eq:11}
\end{equation}

\begin{equation}%eq:eq12
    \begin{aligned}
        \mu x       +   y   + z & = 3   \\
        x           +   2y      & = \mu \\
        (\mu - 1)   +   3y  + z & = 2
    \end{aligned}\label{eq:12}
\end{equation}

\newpage
\section{Απάντηση}

\subsection{Πρόγραμμα}

Το \eqref{eq:11} γράφεται ως \eqref{eq:13} στην μορφή $A\bar{x}=\bar{b}$. Ομοίως, το \eqref{eq:12} γράφεται ως \eqref{eq:14},  με $\mu = 0$ για αριθμό μητρώου 5000.

\begin{equation}%eq:eq13
    \begin{aligned}
        \underbrace{
            \begin{pmatrix}
                3 & 4  & 2 \\
                1 & -2 & 0 \\
                0 & 3  & 1
            \end{pmatrix}
        }_\text{A}
        \underbrace{
            \begin{pmatrix}
                x \\
                y \\
                z
            \end{pmatrix}
        }_\text{$\bar{x}$}
         & =
        \underbrace{
            \begin{pmatrix}
                3 \\
                3 \\
                2
            \end{pmatrix}
        }_\text{$\bar{b}$}
    \end{aligned}\label{eq:13}
\end{equation}

\begin{equation}%eq:eq14
    \begin{aligned}
        \underbrace{
            \begin{pmatrix}
                0  & 1 & 1 \\
                1  & 2 & 0 \\
                -1 & 3 & 1
            \end{pmatrix}
        }_\text{A}
        \underbrace{
            \begin{pmatrix}
                x \\
                y \\
                z
            \end{pmatrix}
        }_\text{$\bar{x}$}
         & =
        \underbrace{
            \begin{pmatrix}
                3 \\
                0 \\
                2
            \end{pmatrix}
        }_\text{$\bar{b}$}
    \end{aligned}\label{eq:14}
\end{equation}

Η έξοδος του προγράμματός μου για τα συστήματα \eqref{eq:13} και \eqref{eq:14}:
\lstinputlisting[language={}]{task1.txt}

\newpage
Το κύριο πρόγραμμα:
\lstinputlisting{task1.m}

\newpage
Η συνάρτηση function\_task1:\label{function_task1}
\lstinputlisting{function_task1.m}

\chapter{Άσκηση 2}
\section{Εκφώνηση}

Φτιάξτε ένα πρόγραμμα για την αντιστροφή ενός πίνακα $A$ με την μέθοδο Gauss, προσαρτώντας δηλαδή τον ταυτοτικό πίνακα Α και ακολουθώντας τον αλγόριθμο της απαλοιφής Gauss. Εξηγήστε γιατί με αυτόν τον τρόπο παίρνετε τον αντίστροφο πίνακα. Το πρόγραμμά σας πρέπει να είναι σε θέση να βρίσκει αν ο πίνακας $A$ είναι αντιστρέψιμος. Εφαρμόστε το πρόγραμμά σας στον πίνακα \eqref{21}.

\begin{equation}
    \begin{aligned}
        A = \begin{pmatrix}
            1  & 2 & 1   \\
            2  & 1 & 0   \\
            -1 & 1 & \mu
        \end{pmatrix}\label{21}
    \end{aligned}
\end{equation}


\newpage
\section{Απάντηση}

\subsection{Πρόγραμμα}

Ο πίνακας \eqref{21} γράφεται ως \eqref{22} για $\mu = 0$.

\begin{equation}
    \begin{aligned}
        A = \begin{pmatrix}
            1  & 2 & 1 \\
            2  & 1 & 0 \\
            -1 & 1 & 0
        \end{pmatrix}\label{22}
    \end{aligned}
\end{equation}

Η έξοδος του προγράμματός μου για τον πίνακα \eqref{22}:
\lstinputlisting[language={}]{task2.txt}

Το πρόγραμμα:
\lstinputlisting{task2.m}


\chapter{Άσκηση 3}
\section{Εκφώνηση}

Θεωρήστε τα διανύσματα του \eqref{31}, όπου $\mu$ είναι το τελευταίο ψηφίο του αριθμού μητρώου σας. Ελέγξτε με το χέρι αν τα παραπάνω διανύσματα είναι γραμμικά ανεξάρτητα ή όχι χωρίς την χρήση του κριτηρίου της ορίζουσας. Αποτελούν αυτά τα διανύσματα βάση του $R^{3}$; Κατόπιν φτιάξτε ένα πρόγραμμα που να αποφασίζει αν μία οποιαδήποτε τριάδα διανυσμάτων στο $R^{3}$ αποτελεί βάση του $R^{3}$ κάνοντας χρήση του κριτηρίου της ορίζουσας.

\begin{equation}%eq:eq31
    \begin{aligned}
        u & = 3i - 4j + 5k  \\
        v & = 2i - 3j + k   \\
        w & = i - j + \mu k
    \end{aligned}\label{31}
\end{equation}

\newpage
\section{Απάντηση}

\subsection{Έλεγχος με το χέρι}
\subsubsection{Στο χέρι}

Το σύστημα διανυσμάτων \eqref{31} γράφεται ως το \eqref{32}, για $\mu = 0$.

\begin{equation}%eq:eq32
    \begin{aligned}
        u & = 3i - 4j + 5k \\
        v & = 2i - 3j + k  \\
        w & = i - j + 0k
    \end{aligned}\label{32}
\end{equation}

Από το \eqref{32} μπορώ να πάρω την σχέση:
\begin{equation}%eq:eq33
    \begin{aligned}
        \lambda_{1}\bar{u} + \lambda_{2}\bar{v} + \lambda_{3}\bar{w} = \bar{0}
    \end{aligned}\label{33}
\end{equation}

Και για  να είναι γραμμικά ανεξάρτητα τα $\bar{u}$, $\bar{v}$, $\bar{w}$ πρέπει να ισχύει $\lambda_{1}=0$, $\lambda_{2}=0$, $\lambda_{3}=0$.

\begin{equation}%eq:eq34
    \begin{aligned}
        \lambda_{1}\begin{pmatrix}
            3 \\ -4 \\ 5
        \end{pmatrix} +
        \lambda_{2}\begin{pmatrix}
            2 \\ -3 \\ 1
        \end{pmatrix} +
        \lambda_{3}\begin{pmatrix}
            1 \\ -1 \\ 0
        \end{pmatrix}        & =
        \begin{pmatrix}
            0 \\ 0 \\ 0
        \end{pmatrix}
        \implies                                         \\
        \underbrace{
            \begin{pmatrix}
                3 & -4 & 5 \\
                2 & -3 & 1 \\
                1 & -1 & 0
            \end{pmatrix}}_\text{A}
        \underbrace{
        \begin{pmatrix}
                \lambda_{1} \\
                \lambda_{2} \\
                \lambda_{3}
            \end{pmatrix}}_\text{$\bar{x}$} & =
        \underbrace{
            \begin{pmatrix}
                0 \\
                0 \\
                0
            \end{pmatrix}}_\text{$\bar{b}$}
    \end{aligned}\label{34}
\end{equation}

Με γραμμοπράξεις μετατρέπω τον πίνακα A του \eqref{34} σε ανηγμένο κλιμακωτό:

\begin{equation*}
    \begin{aligned}
        \begin{pmatrix}
            3 & -4 & 5 \\[0.4cm]
            2 & -3 & 1 \\[0.4cm]
            1 & -1 & 0
        \end{pmatrix}\rowops{,R_{2}-(2/3)R_{1},R_{3}-(1/3)R_{1}}
        \begin{pmatrix}
            3 & -4           & 5            \\[0.4cm]
            0 & -\frac{1}{3} & -\frac{7}{3} \\[0.4cm]
            0 & \frac{1}{3}  & -\frac{5}{3}
        \end{pmatrix}\rowops{,,R_{3}+R_{2}}
        \begin{pmatrix}
            3 & -4           & 5            \\[0.4cm]
            0 & -\frac{1}{3} & -\frac{7}{3} \\[0.4cm]
            0 & 0            & -4
        \end{pmatrix}\rowops{,R_{2}+(7/3)R_{2},R_{3}-(1/4)R_{3}} \\[0.4cm]
        \begin{pmatrix}
            3 & -4           & 5 \\[0.4cm]
            0 & -\frac{1}{3} & 0 \\[0.4cm]
            0 & 0            & 1
        \end{pmatrix}\rowops{R_{1}-5R_{3},-3R_{2},}
        \begin{pmatrix}
            3 & -4 & 0 \\[0.4cm]
            0 & 1  & 0 \\[0.4cm]
            0 & 0  & 1
        \end{pmatrix}
        \rowops{R_{1}+4R_{2},,}
        \begin{pmatrix}
            3 & 0 & 0 \\[0.4cm]
            0 & 1 & 0 \\[0.4cm]
            0 & 0 & 1
        \end{pmatrix}
        \rowops{(1/3)R_{1},,}
        \begin{pmatrix}
            1 & 0 & 0 \\[0.4cm]
            0 & 1 & 0 \\[0.4cm]
            0 & 0 & 1
        \end{pmatrix}
    \end{aligned}
\end{equation*}

Είναι:
\begin{equation*}
    \begin{aligned}
        \underbrace{
            \begin{pmatrix}
                1 & 0 & 0 \\
                0 & 1 & 0 \\
                0 & 0 & 1
            \end{pmatrix}}_\text{A}
        \underbrace{
        \begin{pmatrix}
                \lambda_{1} \\
                \lambda_{2} \\
                \lambda_{3}
            \end{pmatrix}}_\text{$\bar{x}$} & =
        \underbrace{
            \begin{pmatrix}
                0 \\
                0 \\
                0
            \end{pmatrix}}_\text{$\bar{b}$}
        \implies
        \begin{pmatrix}
            \lambda_{1} \\
            \lambda_{2} \\
            \lambda_{3}
        \end{pmatrix}                   & =
        \begin{pmatrix}
            0 \\
            0 \\
            0
        \end{pmatrix}
    \end{aligned}
\end{equation*}

Άρα όντως είναι $\lambda_{1}=0$, $\lambda_{2}=0$, $\lambda_{3}=0$, οπότε και τα $\bar{u}$, $\bar{v}$, $\bar{w}$ είναι γραμικώς ανεξάρτητα στο $R^{3}$.

\subsubsection{Στην Octave}

Για να είναι γραμμικά ανεξάρτητα τα $\bar{u}$, $\bar{v}$, $\bar{w}$ πρέπει να ισχύει η σχέση από το \eqref{33}:
\begin{equation*}
    \lambda_{1} = \lambda_{2} = \lambda_{3} = 0
\end{equation*}

Από το γραμμικό σύστημα εξισώσεων \eqref{34}:

\begin{equation}
    \begin{aligned}%eq:eq35
        \underbrace{
            \begin{pmatrix}
                3 & -4 & 5 \\
                2 & -3 & 1 \\
                1 & -1 & 0
            \end{pmatrix}
        }_\text{A}
        \underbrace{
            \begin{pmatrix}
                \lambda_{1} \\
                \lambda_{2} \\
                \lambda_{3}
            \end{pmatrix}
        }_\text{$\bar{x}$}
         & =
        \underbrace{
            \begin{pmatrix}
                0 \\
                0 \\
                0
            \end{pmatrix}
        }_\text{$\bar{b}$}
    \end{aligned}\label{35}
\end{equation}

Εφαρμόζω στο \eqref{35} την συνάρτηση function\_task1\footnote{(βλ. σελίδα \pageref{function_task1})} από την Άσκηση 1 και δίνει έξοδο:
\lstinputlisting[language={}]{out_task3.txt}
Έτσι, με $\lambda_{1} = 0$, $\lambda_{2} = 0$, $\lambda_{3} = 0$ τα $\bar{u}$, $\bar{v}$, $\bar{w}$ είναι γραμμικά ανεξάρτητα στο $R^{3}$.

Το πρόγραμμα:
\lstinputlisting{out_task3.m}
\newpage

\subsection{Πρόγραμμα}

Για να είναι ένα σετ διανυσμάτων γραμμικά ανεξάρτητο πρέπει η ορίζουσα του πίνακα που προκύπτει, με τα διανύματα αυτά να είναι στήλες του, να έχει τιμή διάφορη του μηδενός. Αλλίως, αυτο το σετ διανυσμάτων είναι γραμμικά εξαρτημένο.

Το κύριο πρόγραμμα:
\lstinputlisting{task3.m}

Η συνάρτηση function\_task3:
\lstinputlisting{function_task3.m}

Η έξοδος του προγράμματός μου:
\lstinputlisting[language={}]{task3.txt}



\chapter{Άσκηση 4}
\section{Εκφώνηση}

Θεωρήστε τα διανύσματα του \eqref{41}. Δείξτε ότι αυτά τα διανύσματα αποτελούν βάση του $R^{3}$. Εφαρμόστε με το χέρι τον αλγόριθμο Gram-Smith για την ορθοκανονικοποίηση της βάσης. Φτιάξτε ένα πρόγραμμα που θα δέχεται σαν είσοδο τρία διανύματα σαν στήλες ενός $3x3$ πίνακα, θα ελέγχει αν αυτά τα διανύσματα αποτελούν βάση του $R^{3}$, και κατόπιν θα εφαρμόζει τον αλγόριθμο Gram-Smith και θα δίνει σαν έξοδο μία ορθοκανονική βάση.

\begin{equation}
    \begin{aligned}
        u & = i - 2j + 3k \\
        v & = 2i - j + 2k \\
        w & = i - 2j + k
    \end{aligned}\label{41}
\end{equation}

\newpage
\section{Απάντηση}

\chapter{Πηγές}
\newpage


\section{Βιβλία}
\section{Video}
\begin{itemize}
    \item \href{https://www.youtube.com/watch?v=cJg2AuSFdjw}{Inverse Matrix Using Gauss-Jordan}
    \item \href{https://youtu.be/X5rDjbp0t6s}{Solving a 3x3 System Using Cramer's Rule}
    \item \href{https://youtu.be/0vB1sgebS9c}{Εύρεση αντίστροφου πίνακα με μέθοδο Gauss}
    \item \href{https://youtu.be/Gmt1fmlrEto}{Γραμμική εξάρτηση διανυσμάτων}
    \item \href{https://youtu.be/yLi8RxqfowA}{Linear Independence and Linear Dependence, Ex 1}
\end{itemize}
\section{Σύνδεσμοι}
\begin{itemize}
    \item \href{http://sites.science.oregonstate.edu/math/home/programs/undergrad/CalculusQuestStudyGuides/vcalc/system/system.html}{Systems of Linear Equations}
    \item \href{http://esperia.iesl.forth.gr/~kafesaki/Applied-Mathematics/linear-algebra/linear-systems.pdf}{Γραμμικά συστήματα Εξισώσεων}
    \item \href{http://www-h.eng.cam.ac.uk/help/programs/octave/tutorial/}{11. Solving Ax = b, Matrix division and the slash operator}
    \item \href{https://onlinemschool.com/math/assistance/equation/gaus/}{Gaussian elimination calculator}
    \item \href{http://www.matrixlab-examples.com/cramers-rule.html}{Cramer's Rule}
    \item \href{https://pt.overleaf.com/learn/latex/Code_listing}{Inserting code in a LaTeX document}
\end{itemize}
\section{Χρήσιμα αρχεία}
\begin{itemize}
    \item \href{http://www.yanivplan.com/files/tutorial2vectors.pdf}{Octave, Vectors and Matrices}
    \item \href{http://web.mit.edu/rsi/www/pdfs/advmath.pdf}{Hardcore LaTeX Math}
\end{itemize}

\end{document}