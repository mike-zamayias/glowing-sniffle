\documentclass[12pt, fleqn, leqno]{extreport}
%\documentclass[12pt]{extreport}
\usepackage[margin=2cm,includeheadfoot,a4paper]{geometry}
\usepackage{fontspec}
%\usepackage[utf8x]{inputenc}
\usepackage[english,greek]{babel}
\usepackage{indentfirst}
\usepackage[dvipsnames]{xcolor}
\usepackage{listings}
\usepackage{titlesec}
\usepackage{amsmath, mathtools}
\usepackage{xifthen, xparse}
\usepackage{fancyhdr}
\usepackage{fancyvrb}
\usepackage{comment}
\usepackage[hyphens]{url}
\usepackage{hyperref}

%\setlength{\mathindent}{0pt}
\setlength{\headheight}{17pt}

%\renewcommand{\arraystretch}{1.5}

\titleformat{\chapter}[display]
  {\normalfont\bfseries}{}{0pt}{\Huge}

\setlength{\parskip}{0cm}
\setlength{\parindent}{1cm}

\setmainfont{[EBGaramond-Regular.ttf]}
\setmonofont{[FiraMono-Regular.otf]}

\hypersetup{
    colorlinks = true,
    linkcolor=black,
    filecolor=magenta,
    urlcolor=blue,
    pdftitle={Project 2}
}


%\definecolor{name}{model}{color-spec}

\lstdefinestyle{mystyle}{
    language=Octave,
    backgroundcolor=\color{white},   
    commentstyle=\color{teal},
    keywordstyle=\color{blue},
    numberstyle=\color{gray}\ttfamily,
    stringstyle=\color{orange},
    basicstyle=\ttfamily\footnotesize,
    breakatwhitespace=false,         
    breaklines=true,                 
    captionpos=b,                    
    keepspaces=true,                 
    numbers=left,                    
    numbersep=5pt,                  
    showspaces=false,                
    showstringspaces=false,
    showtabs=false,                  
    tabsize=2,
    frame=lines,
    framesep=0.1cm,
    rulecolor=\color{black},
    morestring=[b]"
}

\lstset{style=mystyle}

\pagenumbering{arabic}

\pagestyle{fancy}
\fancyhf{}
\fancyhead[R]{\rightmark}
\lhead{Project 2}
\chead{Αριθμητική Γραμμική Άλγεβρα}
\cfoot{\thepage}


\newcommand\rowop[1]{\scriptstyle\smash{\xrightarrow[\vphantom{#1}]{\mkern-4mu#1\mkern-4mu}}}

\DeclareDocumentCommand\converttorows
{>{\SplitList{,}}m}
{\ProcessList{#1}{\converttorow}}
\NewDocumentCommand{\converttorow}{m}
{\ifthenelse{\isempty{#1}}{}{\rowop{#1}}\\}

\DeclareDocumentCommand \rowops{m}
{\;
 \begin{matrix}
\converttorows {#1}
 \end{matrix}
 \; }


\begin{document}

\title{Αριθμητική Γραμμική Άλγεβρα\\Project 2}
\author{Μιχαήλ Ανάργυρος Ζαμάγιας\\TP5000}
\date{\today}
\maketitle
\newpage

\tableofcontents

\chapter{Άσκηση 1}
\section{Ερώτημα}

Δίνεται το σύστημα \eqref{eq:11}. Γράψτε αυτό το σύστημα στην μορφή $A\bar{x}=\bar{b}$ για κατάλληλα διανύσματα $\bar{x}$ και $\bar{b}$. Εδώ $\mu$ είναι το τελευταίο ψηφίο του αριθμού μητρώου σας. Φτιάξτε ένα πρόγραμμα που να δέχεται ως δεδομένα έναν $3x3$ πίνακα $A$ και ένα $3x1$ διάνυσμα $\bar{b}$ και να είναι σε θεση να μας πει αν το σύστημα είναι αόριστο ή αδύνατο ή έχει μοναδική λύση. Αν το σύστημα έχει μοναδική λύση πρέπει να δίνει αυτή την λύση υπολογισμένη με την μέθοδο του επαυξημένου πίνακα (απαλοιφή Gauss). Δώστε τα αποτελέσματα του προγράμματός σας στο σύστημα εξισώσεων \eqref{eq:11} καθώς και στο σύστημα \eqref{eq:12}.

\begin{equation}%eq11
    \begin{aligned}
        3x      +   4y    + 2z & = 3 \\
        x       -   2y         & = 3 \\
        \mu x   +   3y    + z  & = 2
    \end{aligned}\label{eq:11}
\end{equation}

\begin{equation}%eq:eq12
    \begin{aligned}
        \mu x       +   y   + z & = 3   \\
        x           +   2y      & = \mu \\
        (\mu - 1)   +   3y  + z & = 2
    \end{aligned}\label{eq:12}
\end{equation}

\newpage
\section{Λύση}

\subsection{Πρόγραμμα}

Το \eqref{eq:11} γράφεται ως \eqref{eq:13} στην μορφή $A\bar{x}=\bar{b}$. Ομοίως, το \eqref{eq:12} γράφεται ως \eqref{eq:14},  με $\mu = 0$ για αριθμό μητρώου 5000.

\begin{equation}%eq:eq13
    \begin{aligned}
        \underbrace{
            \begin{pmatrix}
                3 & 4  & 2 \\
                1 & -2 & 0 \\
                0 & 3  & 1
            \end{pmatrix}
        }_\text{A}
        \underbrace{
            \begin{pmatrix}
                x \\
                y \\
                z
            \end{pmatrix}
        }_\text{$\bar{x}$}
         & =
        \underbrace{
            \begin{pmatrix}
                3 \\
                3 \\
                2
            \end{pmatrix}
        }_\text{$\bar{b}$}
    \end{aligned}\label{eq:13}
\end{equation}

\begin{equation}%eq:eq14
    \begin{aligned}
        \underbrace{
            \begin{pmatrix}
                0  & 1 & 1 \\
                1  & 2 & 0 \\
                -1 & 3 & 1
            \end{pmatrix}
        }_\text{A}
        \underbrace{
            \begin{pmatrix}
                x \\
                y \\
                z
            \end{pmatrix}
        }_\text{$\bar{x}$}
         & =
        \underbrace{
            \begin{pmatrix}
                3 \\
                0 \\
                2
            \end{pmatrix}
        }_\text{$\bar{b}$}
    \end{aligned}\label{eq:14}
\end{equation}

Η έξοδος του προγράμματός μου για τα συστήματα \eqref{eq:13} και \eqref{eq:14}:
\lstinputlisting[language={}]{task1.txt}

\newpage
Το κύριο πρόγραμμα:
\lstinputlisting{task1.m}

\newpage
Η συνάρτηση lin\_sys:\label{lin_sys}
\lstinputlisting{lin_sys.m}

\chapter{Άσκηση 2}
\section{Ερώτημα}

Φτιάξτε ένα πρόγραμμα για την αντιστροφή ενός πίνακα $A$ με την μέθοδο Gauss, προσαρτώντας δηλαδή τον ταυτοτικό πίνακα Α και ακολουθώντας τον αλγόριθμο της απαλοιφής Gauss. Εξηγήστε γιατί με αυτόν τον τρόπο παίρνετε τον αντίστροφο πίνακα. Το πρόγραμμά σας πρέπει να είναι σε θέση να βρίσκει αν ο πίνακας $A$ είναι αντιστρέψιμος. Εφαρμόστε το πρόγραμμά σας στον πίνακα \eqref{21}.

\begin{equation}
    \begin{aligned}
        A = \begin{pmatrix}
            1  & 2 & 1   \\
            2  & 1 & 0   \\
            -1 & 1 & \mu
        \end{pmatrix}\label{21}
    \end{aligned}
\end{equation}


\newpage
\section{Λύση}

\subsection{Πρόγραμμα}

Ο πίνακας \eqref{21} γράφεται ως \eqref{22} για $\mu = 0$.

\begin{equation}
    \begin{aligned}
        A = \begin{pmatrix}
            1  & 2 & 1 \\
            2  & 1 & 0 \\
            -1 & 1 & 0
        \end{pmatrix}\label{22}
    \end{aligned}
\end{equation}

Η έξοδος του προγράμματός μου για τον πίνακα \eqref{22}:
\lstinputlisting[language={}]{task2.txt}

Το πρόγραμμα:
\lstinputlisting{task2.m}


\chapter{Άσκηση 3}
\section{Ερώτημα}

Θεωρήστε τα διανύσματα του \eqref{31}, όπου $\mu$ είναι το τελευταίο ψηφίο του αριθμού μητρώου σας. Ελέγξτε με το χέρι αν τα παραπάνω διανύσματα είναι γραμμικά ανεξάρτητα ή όχι χωρίς την χρήση του κριτηρίου της ορίζουσας. Αποτελούν αυτά τα διανύσματα βάση του $R^{3}$; Κατόπιν φτιάξτε ένα πρόγραμμα που να αποφασίζει αν μία οποιαδήποτε τριάδα διανυσμάτων στο $R^{3}$ αποτελεί βάση του $R^{3}$ κάνοντας χρήση του κριτηρίου της ορίζουσας.

\begin{equation}%eq:eq31
    \begin{aligned}
        u & = 3i - 4j + 5k  \\
        v & = 2i - 3j + k   \\
        w & = i - j + \mu k
    \end{aligned}\label{31}
\end{equation}

\newpage
\section{Λύση}

\subsection{Έλεγχος με το χέρι}
\subsubsection{Στο χέρι}

Το σύστημα διανυσμάτων \eqref{31} γράφεται ως το \eqref{32}, για $\mu = 0$.

\begin{equation}%eq:eq32
    \begin{aligned}
        u & = 3i - 4j + 5k \\
        v & = 2i - 3j + k  \\
        w & = i - j + 0k
    \end{aligned}\label{32}
\end{equation}

Από το \eqref{32} μπορώ να πάρω την σχέση:
\begin{equation}%eq:eq33
    \begin{aligned}
        \lambda_{1}\bar{u} + \lambda_{2}\bar{v} + \lambda_{3}\bar{w} = \bar{0}
    \end{aligned}\label{33}
\end{equation}

Και για  να είναι γραμμικά ανεξάρτητα τα $\bar{u}$, $\bar{v}$, $\bar{w}$ πρέπει να ισχύει $\lambda_{1}=0$, $\lambda_{2}=0$, $\lambda_{3}=0$.

\begin{equation}%eq:eq34
    \begin{aligned}
        \lambda_{1}\begin{pmatrix}
            3 \\ -4 \\ 5
        \end{pmatrix} +
        \lambda_{2}\begin{pmatrix}
            2 \\ -3 \\ 1
        \end{pmatrix} +
        \lambda_{3}\begin{pmatrix}
            1 \\ -1 \\ 0
        \end{pmatrix}        & =
        \begin{pmatrix}
            0 \\ 0 \\ 0
        \end{pmatrix}
        \implies                                         \\
        \underbrace{
            \begin{pmatrix}
                3 & -4 & 5 \\
                2 & -3 & 1 \\
                1 & -1 & 0
            \end{pmatrix}}_\text{A}
        \underbrace{
        \begin{pmatrix}
                \lambda_{1} \\
                \lambda_{2} \\
                \lambda_{3}
            \end{pmatrix}}_\text{$\bar{x}$} & =
        \underbrace{
            \begin{pmatrix}
                0 \\
                0 \\
                0
            \end{pmatrix}}_\text{$\bar{b}$}
    \end{aligned}\label{34}
\end{equation}

Με γραμμοπράξεις μετατρέπω τον πίνακα A του \eqref{34} σε ανηγμένο κλιμακωτό:

\begin{equation*}
    \begin{aligned}
        \begin{pmatrix}
            3 & -4 & 5 \\[0.2cm]
            2 & -3 & 1 \\[0.2cm]
            1 & -1 & 0
        \end{pmatrix}\rowops{,R_{2}-(2/3)R_{1},R_{3}-(1/3)R_{1}}
        \begin{pmatrix}
            3 & -4           & 5            \\[0.2cm]
            0 & -\frac{1}{3} & -\frac{7}{3} \\[0.2cm]
            0 & \frac{1}{3}  & -\frac{5}{3}
        \end{pmatrix}\rowops{,,R_{3}+R_{2}}
        \begin{pmatrix}
            3 & -4           & 5            \\[0.2cm]
            0 & -\frac{1}{3} & -\frac{7}{3} \\[0.2cm]
            0 & 0            & -4
        \end{pmatrix}\rowops{,R_{2}+(7/3)R_{2},R_{3}-(1/4)R_{3}} \\[0.2cm]
        \begin{pmatrix}
            3 & -4           & 5 \\[0.2cm]
            0 & -\frac{1}{3} & 0 \\[0.2cm]
            0 & 0            & 1
        \end{pmatrix}\rowops{R_{1}-5R_{3},-3R_{2},}
        \begin{pmatrix}
            3 & -4 & 0 \\[0.2cm]
            0 & 1  & 0 \\[0.2cm]
            0 & 0  & 1
        \end{pmatrix}
        \rowops{R_{1}+4R_{2},,}
        \begin{pmatrix}
            3 & 0 & 0 \\[0.2cm]
            0 & 1 & 0 \\[0.2cm]
            0 & 0 & 1
        \end{pmatrix}
        \rowops{(1/3)R_{1},,}
        \begin{pmatrix}
            1 & 0 & 0 \\[0.2cm]
            0 & 1 & 0 \\[0.2cm]
            0 & 0 & 1
        \end{pmatrix}
    \end{aligned}
\end{equation*}

Είναι:
\begin{equation*}
    \begin{aligned}
        \underbrace{
            \begin{pmatrix}
                1 & 0 & 0 \\
                0 & 1 & 0 \\
                0 & 0 & 1
            \end{pmatrix}}_\text{A}
        \underbrace{
        \begin{pmatrix}
                \lambda_{1} \\
                \lambda_{2} \\
                \lambda_{3}
            \end{pmatrix}}_\text{$\bar{x}$} & =
        \underbrace{
            \begin{pmatrix}
                0 \\
                0 \\
                0
            \end{pmatrix}}_\text{$\bar{b}$}
        \implies
        \begin{pmatrix}
            \lambda_{1} \\
            \lambda_{2} \\
            \lambda_{3}
        \end{pmatrix}                   & =
        \begin{pmatrix}
            0 \\
            0 \\
            0
        \end{pmatrix}
    \end{aligned}
\end{equation*}

Άρα όντως είναι $\lambda_{1}=0$, $\lambda_{2}=0$, $\lambda_{3}=0$, οπότε και τα $\bar{u}$, $\bar{v}$, $\bar{w}$ είναι γραμικώς ανεξάρτητα στο $R^{3}$.

\subsubsection{Στην Octave}

Για να είναι γραμμικά ανεξάρτητα τα $\bar{u}$, $\bar{v}$, $\bar{w}$ πρέπει να ισχύει η σχέση από το \eqref{33}:
\begin{equation*}
    \lambda_{1} = \lambda_{2} = \lambda_{3} = 0
\end{equation*}

Από το γραμμικό σύστημα εξισώσεων \eqref{34}:

\begin{equation}
    \begin{aligned}%eq:eq35
        \underbrace{
            \begin{pmatrix}
                3 & -4 & 5 \\
                2 & -3 & 1 \\
                1 & -1 & 0
            \end{pmatrix}
        }_\text{A}
        \underbrace{
            \begin{pmatrix}
                \lambda_{1} \\
                \lambda_{2} \\
                \lambda_{3}
            \end{pmatrix}
        }_\text{$\bar{x}$}
         & =
        \underbrace{
            \begin{pmatrix}
                0 \\
                0 \\
                0
            \end{pmatrix}
        }_\text{$\bar{b}$}
    \end{aligned}\label{35}
\end{equation}

Εφαρμόζω στο \eqref{35} την συνάρτηση lin\_sys\footnote{Η συνάρτηση lin\_sys στην σελίδα \pageref{lin_sys}.} από την Άσκηση 1 και δίνει έξοδο:
\lstinputlisting[language={}]{out_task3.txt}
Έτσι, με $\lambda_{1} = 0$, $\lambda_{2} = 0$, $\lambda_{3} = 0$ τα $\bar{u}$, $\bar{v}$, $\bar{w}$ είναι γραμμικά ανεξάρτητα στο $R^{3}$.

Το πρόγραμμα:
\lstinputlisting{out_task3.m}
\newpage

\subsection{Πρόγραμμα}

Ένα σετ $m$ διανυσμάτων μήκους $n$ είναι γραμμικά ανεξάρτητο όταν η ορίζουσα του πίνακα, που προκύπτει με τα διανύματα αυτά να είναι στήλες του, να είναι μη μηδενική. Αλλίως, αυτο το σετ διανυσμάτων είναι γραμμικά εξαρτημένο.

Το κύριο πρόγραμμα:
\lstinputlisting{task3.m}

Η συνάρτηση lin\_dep:
\lstinputlisting{lin_dep.m}

Η έξοδος του προγράμματός μου:
\lstinputlisting[language={}]{task3.txt}

\chapter{Άσκηση 4}
\section{Ερώτημα}

Θεωρήστε τα διανύσματα του \eqref{41}. Δείξτε ότι αυτά τα διανύσματα αποτελούν βάση του $R^{3}$. Εφαρμόστε με το χέρι τον αλγόριθμο Gram-Smith για την ορθοκανονικοποίηση της βάσης. Φτιάξτε ένα πρόγραμμα που θα δέχεται σαν είσοδο τρία διανύματα σαν στήλες ενός $3x3$ πίνακα, θα ελέγχει αν αυτά τα διανύσματα αποτελούν βάση του $R^{3}$, και κατόπιν θα εφαρμόζει τον αλγόριθμο Gram-Smith και θα δίνει σαν έξοδο μία ορθοκανονική βάση.

\begin{equation}
    \begin{aligned}
        u & = i - 2j + 3k \\
        v & = 2i - j + 2k \\
        w & = i - 2j + k
    \end{aligned}\label{41}
\end{equation}

\newpage
\section{Λύση}

\subsection{Έλεγχος αν τα διανύσματα αποτελούν βάση}

\subsubsection{Στο χέρι}

Για να δείξουμε ότι τα διανύσματα 
$
    u = \begin{pmatrix}
        1 \\ -2 \\ 3
    \end{pmatrix},
    v = \begin{pmatrix}
        2 \\ -1 \\ 2
    \end{pmatrix},
    w = \begin{pmatrix}
        1 \\ -2 \\ 1
    \end{pmatrix}
$
αποτελούν βάση του $R^{3}$ αρκεί να δείξουμε ότι τα διανύσματα είναι γραμμικά ανεξάρτητα.

Έχουμε το σύστημα $A\bar{x}=\bar{b}$:
\begin{equation*}
    \begin{aligned}
        \underbrace{
            \begin{pmatrix}
                1  & 2  & 1  \\
                -2 & -1 & -2 \\
                3  & 2  & 1
            \end{pmatrix}}_\text{A}
        \underbrace{
        \begin{pmatrix}
                \lambda_{1} \\
                \lambda_{2} \\
                \lambda_{3}
            \end{pmatrix}}_\text{$\bar{x}$} & =
        \underbrace{
            \begin{pmatrix}
                0 \\
                0 \\
                0
            \end{pmatrix}}_\text{$\bar{b}$}
    \end{aligned}
\end{equation*}

Με γραμμοπράξεις μετατρέπουμε τον πίνακα A σε ανηγμένο κλιμακωτό:

\begin{equation*}
    \begin{aligned}
        \begin{pmatrix}
            1  & 2  & 1  \\[0.2cm]
            -2 & -1 & -2 \\[0.2cm]
            3  & 2  & 1
        \end{pmatrix}\rowops{R_{1}=R_{3},,R_{3}=R_{1}}
        \begin{pmatrix}
            3  & 2  & 1  \\[0.2cm]
            -2 & -1 & -2 \\[0.2cm]
            1  & 2  & 1
        \end{pmatrix}\rowops{,R_{2}=R_{2}+(2/3)R_{1},R_{3}=R_{3}-(1/3)R_{1}}
        \begin{pmatrix}
            3 & 2           & 1            \\[0.2cm]
            0 & \frac{1}{3} & -\frac{4}{3} \\[0.2cm]
            0 & \frac{4}{3} & \frac{2}{3}
        \end{pmatrix}\rowops{,R_{2}=R_{3},R_{3}=R_{2}}                  \\
        \begin{pmatrix}
            3 & 2           & 1            \\[0.2cm]
            0 & \frac{4}{3} & \frac{2}{3}  \\[0.2cm]
            0 & \frac{1}{3} & -\frac{4}{3}
        \end{pmatrix}\rowops{,,R_{3}=R_{3}-(1/4)R_{2}}
        \begin{pmatrix}
            3 & 2           & 1            \\[0.2cm]
            0 & \frac{4}{3} & \frac{2}{3}  \\[0.2cm]
            0 & 0           & -\frac{3}{2}
        \end{pmatrix}\rowops{,,R_{3}=-(2/3)R_{3}}
        \begin{pmatrix}
            3 & 2           & 1           \\[0.2cm]
            0 & \frac{4}{3} & \frac{2}{3} \\[0.2cm]
            0 & 0           & 1
        \end{pmatrix}\rowops{R_{1}=R_{1}-R_{3},R_{2}=R_{2}-(2/3)R_{3},} \\
        \begin{pmatrix}
            3 & 2           & 0 \\[0.2cm]
            0 & \frac{4}{3} & 0 \\[0.2cm]
            0 & 0           & 1
        \end{pmatrix}\rowops{,R_{2}=(3/4)R_{2},}
        \begin{pmatrix}
            3 & 2 & 0 \\[0.2cm]
            0 & 1 & 0 \\[0.2cm]
            0 & 0 & 1
        \end{pmatrix}\rowops{R_{1}=R_{1}-2R_{2},,}
        \begin{pmatrix}
            3 & 0 & 0 \\[0.2cm]
            0 & 1 & 0 \\[0.2cm]
            0 & 0 & 1
        \end{pmatrix}\rowops{R_{1}=(1/3)R_{1},,}
        \begin{pmatrix}
            1 & 0 & 0 \\[0.2cm]
            0 & 1 & 0 \\[0.2cm]
            0 & 0 & 1
        \end{pmatrix}
    \end{aligned}
\end{equation*}

Οπότε είναι:
\begin{equation*}
    \begin{aligned}
        \underbrace{
            \begin{pmatrix}
                1 & 0 & 0 \\
                0 & 1 & 0 \\
                0 & 0 & 1
            \end{pmatrix}}_\text{A}
        \underbrace{
        \begin{pmatrix}
                \lambda_{1} \\
                \lambda_{2} \\
                \lambda_{3}
            \end{pmatrix}}_\text{$\bar{x}$} & =
        \underbrace{
            \begin{pmatrix}
                0 \\
                0 \\
                0
            \end{pmatrix}}_\text{$\bar{b}$}
        \implies
        \begin{pmatrix}
            \lambda_{1} \\
            \lambda_{2} \\
            \lambda_{3}
        \end{pmatrix}                   & =
        \begin{pmatrix}
            0 \\
            0 \\
            0
        \end{pmatrix}
    \end{aligned}
\end{equation*}

Είναι $\lambda_{1}=0$, $\lambda_{2}=0$, $\lambda_{3}=0$, επομένως και τα $\bar{u}$, $\bar{v}$, $\bar{w}$ είναι γραμικώς ανεξάρτητα στο $R^{3}$, άρα αποτελούν βάση του $R^{3}$.

\newpage
\subsubsection{Στην Octave}

Δίνω τα διανύσματα
$
    u = \begin{pmatrix}
        1 \\ -2 \\ 3
    \end{pmatrix},
    v = \begin{pmatrix}
        2 \\ -1 \\ 2
    \end{pmatrix},
    w = \begin{pmatrix}
        1 \\ -2 \\ 1
    \end{pmatrix}
$
στο πρόγραμμα της Άσκησης 3. Η έξοδος του προγράμματoς:
\lstinputlisting[language={}]{task3.txt}
Εφόσον τα $\bar{u}$, $\bar{v}$, $\bar{w}$ είναι γραμικώς ανεξάρτητα στο $R^{3}$, αποτελούν βάση του $R^{3}$.

Το πρόγραμμα:
\lstinputlisting{task3.m}
Η συνάρτηση lin\_dep:
\lstinputlisting{lin_dep.m}


\newpage
\subsection{Ορθοκανονικοποίηση βάσης με τον αλγόριθμο Gram-Smith}

Έστω η βάση $S=[u, v, w]$ του $R_{3}$, αφού $u, v, w$ γραμμικά ανεξάρτητα.

Κατασκευάζουμε την ορθογώνια βάση $S'=[a, b, c]$:

%   orthogonal a
\begin{equation}
    \begin{split}
        a &= u
        \\&=
        \begin{pmatrix}
            1 & -2 & 3
        \end{pmatrix}
    \end{split}
\end{equation}


%   orthogona
\begin{equation}
    \begin{split}
        b &= v - \frac{a \cdot v}{a \cdot a}a
        \\ &=
        \begin{pmatrix}
            2 & -1 & 2
        \end{pmatrix}
        - \frac{
            \begin{pmatrix}
                1 & -2 & 3
            \end{pmatrix}
            \cdot
            \begin{pmatrix}
                2 & -1 & 2
            \end{pmatrix}
        }{
            \begin{pmatrix}
                1 & -2 & 3
            \end{pmatrix}
            \cdot
            \begin{pmatrix}
                1 & -2 & 3
            \end{pmatrix}
        }
        \\ &=
        \begin{pmatrix}
            1 & -2 & 3
        \end{pmatrix}
        \begin{pmatrix}
            2 & -1 & 2
        \end{pmatrix}
        - \frac{
            \begin{pmatrix}
                1 & -2 & 3
            \end{pmatrix}
            \cdot
            \begin{pmatrix}
                2 & -1 & 2
            \end{pmatrix}
        }{14}
        \begin{pmatrix}
            1 & -2 & 3
        \end{pmatrix}
        \\ &=
        \begin{pmatrix}
            2 & -1 & 2
        \end{pmatrix}
        - \frac{10}{14}
        \begin{pmatrix}
            1 & -2 & 3
        \end{pmatrix}
        \\ &=
        \begin{pmatrix}
            2 & -1 & 2
        \end{pmatrix}
        -
        \begin{pmatrix}
            \frac{5}{7} & -\frac{10}{7} & \frac{15}{7}
        \end{pmatrix}
        \\ &=
        \begin{pmatrix}
            \frac{9}{7} & \frac{3}{7} & -\frac{1}{7}
        \end{pmatrix}
    \end{split}
\end{equation}

%   orthogonal c
\begin{equation}
    \begin{split}
        c & = w - \frac{a \cdot w}{a \cdot a}a - \frac{b \cdot w}{b \cdot b}b
        \\ &=
        \begin{pmatrix}
            1 & -2 & 1
        \end{pmatrix}
        - \frac{
            \begin{pmatrix}
                1 & -2 & 3
            \end{pmatrix}
            \cdot
            \begin{pmatrix}
                1 & -2 & 1
            \end{pmatrix}
        }{14}
        \begin{pmatrix}
            1 & -2 & 3
        \end{pmatrix}
        - \frac{b \cdot w}{b \cdot b}b
        \\ &=
        \begin{pmatrix}
            1 & -2 & 1
        \end{pmatrix}
        - \frac{8}{14}
        \begin{pmatrix}
            1 & -2 & 3
        \end{pmatrix}
        - \frac{b \cdot w}{b \cdot b}b
        \\ &=
        \begin{pmatrix}
            1 & -2 & 1
        \end{pmatrix}
        -
        \begin{pmatrix}
            \frac{4}{7} & -\frac{8}{7} & \frac{12}{7}
        \end{pmatrix}
        - \frac{b \cdot w}{b \cdot b}b
        \\ &=
        \begin{pmatrix}
            \frac{3}{7} & -\frac{6}{7} & -\frac{5}{7}
        \end{pmatrix}
        - \frac{
            \begin{pmatrix}
                \frac{9}{7} & \frac{3}{7} & -\frac{1}{7}
            \end{pmatrix}
            \cdot
            \begin{pmatrix}
                1 & -2 & 1
            \end{pmatrix}
        }{
            \begin{pmatrix}
                \frac{9}{7} & \frac{3}{7} & -\frac{1}{7}
            \end{pmatrix}
            \cdot
            \begin{pmatrix}
                \frac{9}{7} & \frac{3}{7} & -\frac{1}{7}
            \end{pmatrix}
        }
        \begin{pmatrix}
            \frac{9}{7} & \frac{3}{7} & -\frac{1}{7}
        \end{pmatrix}
        \\ &=
        \begin{pmatrix}
            \frac{3}{7} & -\frac{6}{7} & -\frac{5}{7}
        \end{pmatrix}
        - \frac{
            \begin{pmatrix}
                \frac{9}{7} & \frac{3}{7} & -\frac{1}{7}
            \end{pmatrix}
            \cdot
            \begin{pmatrix}
                1 & -2 & 1
            \end{pmatrix}
        }{
            \frac{13}{7}
        }
        \begin{pmatrix}
            \frac{9}{7} & \frac{3}{7} & -\frac{1}{7}
        \end{pmatrix}
        \\ &=
        \begin{pmatrix}
            \frac{3}{7} & -\frac{6}{7} & -\frac{5}{7}
        \end{pmatrix}
        - \frac{
            \frac{2}{7}
        }{
            \frac{13}{7}
        }
        \begin{pmatrix}
            \frac{9}{7} & \frac{3}{7} & -\frac{1}{7}
        \end{pmatrix}
        \\ &=
        \begin{pmatrix}
            \frac{3}{7} & -\frac{6}{7} & -\frac{5}{7}
        \end{pmatrix}
        -
        \begin{pmatrix}
            \frac{18}{91} & \frac{6}{91} & -\frac{2}{91}
        \end{pmatrix}
        \\ &=
        \begin{pmatrix}
            \frac{3}{13} & -\frac{12}{13} & -\frac{9}{13}
        \end{pmatrix}
    \end{split}
\end{equation}

Κατασκευάζουμε την ορθοκανονική βάση $S''=[e_{1}, e_{2}, e_{3}]$:

%   orthonormal e1
\begin{equation}
    \begin{split}
        e_{1} &= \frac{a}{\vert a \vert} =
        \frac{
            \begin{pmatrix}
                1 & -2 & 3
            \end{pmatrix}
        }{
            \vert
            \begin{pmatrix}
                1 & -2 & 3
            \end{pmatrix}
            \vert
        } =
        \frac{
            \begin{pmatrix}
                1 & -2 & 3
            \end{pmatrix}
        }{
            \sqrt{14}
        } =
        \begin{pmatrix}
            \frac{1}{\sqrt{14}} & \frac{-2}{\sqrt{14}} & \frac{3}{\sqrt{14}}
        \end{pmatrix}
    \end{split}
\end{equation}

%   orthonormal e2
\begin{equation}
    \begin{split}
        e_{2} &= \frac{b}{\vert b \vert} =
        \frac{
            \begin{pmatrix}
                \frac{9}{7} & \frac{3}{7} & -\frac{1}{7}
            \end{pmatrix}
        }{
            \vert
            \begin{pmatrix}
                \frac{9}{7} & \frac{3}{7} & -\frac{1}{7}
            \end{pmatrix}
            \vert
        } =
        \frac{
            \begin{pmatrix}
                \frac{9}{7} & \frac{3}{7} & -\frac{1}{7}
            \end{pmatrix}
        }{
            \sqrt{\frac{13}{7}}
        } =
        \begin{pmatrix}
            \frac{9}{\sqrt{91}} & \frac{3}{\sqrt{91}} & -\frac{1}{\sqrt{91}}
        \end{pmatrix}
    \end{split}
\end{equation}

%   orthonormal e3
\begin{equation}
    \begin{split}
        e_{3} &= \frac{c}{\vert c \vert} =
        \frac{
            \begin{pmatrix}
                \frac{3}{13} & -\frac{12}{13} & -\frac{9}{13}
            \end{pmatrix}
        }{
            \vert
            \begin{pmatrix}
                \frac{3}{13} & -\frac{12}{13} & -\frac{9}{13}
            \end{pmatrix}
            \vert
        } =
        \frac{
            \begin{pmatrix}
                \frac{3}{13} & -\frac{12}{13} & -\frac{9}{13}
            \end{pmatrix}
        }{
            \frac{3\sqrt{2}}{\sqrt{13}}
        } =
        \begin{pmatrix}
            \frac{1}{\sqrt{26}} & -\frac{2\sqrt{2}}{\sqrt{13}} & -\frac{3}{\sqrt{26}}
        \end{pmatrix}
    \end{split}
\end{equation}

\newpage
\subsection{Πρόγραμμα}

Το κύριο πρόγραμμα:
\lstinputlisting{task4.m}

Η συνάρτηση orthonorm\_base:
\lstinputlisting{orthonorm_base.m}

Η έξοδος του προγράμματός μου:
\lstinputlisting[language={}]{task4.txt}


\chapter{Πηγές}
\newpage

\section{Βιβλία}
\section{Video}
\begin{itemize}
    \item \href{https://www.youtube.com/watch?v=cJg2AuSFdjw}{Inverse Matrix Using Gauss-Jordan}
    \item \href{https://youtu.be/X5rDjbp0t6s}{Solving a 3x3 System Using Cramer's Rule}
    \item \href{https://youtu.be/0vB1sgebS9c}{Εύρεση αντίστροφου πίνακα με μέθοδο Gauss}
    \item \href{https://youtu.be/Gmt1fmlrEto}{Γραμμική εξάρτηση διανυσμάτων}
    \item \href{https://youtu.be/yLi8RxqfowA}{Linear Independence and Linear Dependence, Ex 1}
    \item \href{https://www.youtube.com/watch?v=ZJu26chXEiw}{Linear Algebra: Orthonormal Basis}
    \item \href{https://www.youtube.com/watch?v=Aslf3KGq2UE}{Linear Algebra: Gram-Schmidt}
\end{itemize}
\section{Σύνδεσμοι}
\begin{itemize}
    \item \href{http://sites.science.oregonstate.edu/math/home/programs/undergrad/CalculusQuestStudyGuides/vcalc/system/system.html}{Systems of Linear Equations}
    \item \href{http://esperia.iesl.forth.gr/~kafesaki/Applied-Mathematics/linear-algebra/linear-systems.pdf}{Γραμμικά συστήματα Εξισώσεων}
    \item \href{http://www-h.eng.cam.ac.uk/help/programs/octave/tutorial/}{11. Solving Ax = b, Matrix division and the slash operator}
    \item \href{https://onlinemschool.com/math/assistance/equation/gaus/}{Gaussian elimination calculator}
    \item \href{http://www.matrixlab-examples.com/cramers-rule.html}{Cramer's Rule}
    \item \href{https://pt.overleaf.com/learn/latex/Code_listing}{Inserting code in a LaTeX document}
    \item \href{http://sites.science.oregonstate.edu/math/home/programs/undergrad/CalculusQuestStudyGuides/vcalc/lindep/lindep.html}{Testing for Linear Dependence of Vectors}
\end{itemize}
\section{Χρήσιμα αρχεία}
\begin{itemize}
    \item \href{http://www.yanivplan.com/files/tutorial2vectors.pdf}{Octave, Vectors and Matrices}
    \item \href{http://web.mit.edu/rsi/www/pdfs/advmath.pdf}{Hardcore LaTeX Math}
    \item \href{http://www.katsetis.gr/mathimataeapsimeioseis/plh12/PLI12bash_kai_diastash_dianysmatikoy_xwroy.pdf}{Βάση και Διάσταση Διανυσματικού Χώρου}
\end{itemize}

\end{document}