\documentclass[a4paper,14pt]{extreport}
\usepackage[margin=2cm,includeheadfoot]{geometry}
\usepackage{fontspec}
\usepackage[utf8x]{inputenc}
\usepackage[english,greek]{babel}
\usepackage{indentfirst}
\usepackage{listings}
\usepackage[hyphens]{url} 
\usepackage{hyperref}
\usepackage{titlesec}
\usepackage{systeme}
\usepackage{amsmath}
\usepackage{fancyhdr}

\titleformat{\chapter}[display]
  {\normalfont\bfseries}{}{0pt}{\Huge}

\setlength{\parskip}{0cm}
\setlength{\parindent}{1cm}

\setmainfont{[EBGaramond-Regular.ttf]}
\setmonofont{[FiraMono-Regular.otf]}

\hypersetup{
    colorlinks = true,
    linkcolor=black,
    filecolor=magenta,      
    urlcolor=blue,
    pdftitle={Project 2}
}

\lstdefinestyle{mystyle}{
    basicstyle=\ttfamily\small,
    breakatwhitespace=false,         
    breaklines=true,                 
    captionpos=b,                    
    keepspaces=true,                 
    numbers=left,                    
    numbersep=5pt,                  
    showspaces=false,                
    showstringspaces=false,
    showtabs=false,                  
    tabsize=4
}
\lstset{style=mystyle}

\pagenumbering{arabic}

\pagestyle{fancy}
\fancyhf{}
\rhead{
    Project 2
    }
\lhead{
    Αριθμητική Γραμμική Άλγεβρα
}
\cfoot{\thepage}

        
\begin{document}

\title{Αριθμητική Γραμμική Άλγεβρα\\Project 2}
\author{Μιχαήλ Ανάργυρος Ζαμάγιας}
\date{\today}
\maketitle
\newpage

\tableofcontents

\chapter{Άσκηση 1}
\section{Εκφώνηση}

Δίνεται το σύστημα (1.1.1). Γράψτε αυτό το σύστημα στην μορφή $Ax=b$ για κατάλληλα διανύσματα $x$ και $b$. Εδώ $\mu$ είναι το τελευταίο ψηφίο του αριθμού μητρώου σας. Φτιάξτε ένα πρόγραμμα που να δέχεται ως δεδομένα έναν $3x3$ πίνακα $A$ και ένα $3x1$ διάνυσμα $b$ και να είναι σε θεση να μας πει αν το σύστημα είναι αόριστο ή αδύνατο ή έχει μοναδική λύση. Αν το σύστημα έχει μοναδική λύση πρέπει να δίνει αυτή την λύση υπολογισμένη με την μέθοδο του επαυξημένου πίνακα (απαλοιφή Gauss). Δώστε τα αποτελέσματα του προγράμματός σας στο σύστημα εξισώσεων (1.1.1) καθώς και στο σύστημα (1.1.2).

\begin{align*}
    \text{(1.1.1)}\systeme[\quadxyz]{
        3x+4y+2z=3,x-2y=3,$\mu$x+3y+z=$\mu$-1
    }
     &  &
    \text{(1.1.2)}\systeme[\quadxyz]{
        $\mu$x+y+z=3,x+2y=$\mu$,($\mu$-1)x+3y+z=2
    }
\end{align*}

\newpage
\section{Απάντηση}

\chapter{Άσκηση 2}
\section{Εκφώνηση}

Φτιάξτε ένα πρόγραμμα για την αντιστροφή ενός πίνακα $A$ με την μέθοδο Gauss, προσαρτώντας δηλαδή τον ταυτοτικό πίνακα Α και ακολουθώντας τον αλγόριθμο της απαλοιφής Gauss. Εξηγήστε γιατί με αυτόν τον τρόπο παίρνετε τον αντίστροφο πίνακα. Το πρόγραμμά σας πρέπει να είναι σε θέση να βρίσκει αν ο πίνακας $A$ είναι αντιστρέψιμος. Εφαρμόστε το πρόγραμμά σας στον πίνακα (2.1.1).
\begin{align*}
    \text{(2.1.1)}\systeme{
        A = \begin{pmatrix}
            1  & 2 & 1   \\
            2  & 1 & 0   \\
            -1 & 1 & \mu
        \end{pmatrix}
    }
\end{align*}

\newpage
\section{Απάντηση}

\chapter{Άσκηση 3}
\section{Εκφώνηση}

Θεωρήστε τα διανύσματα του (3.1.1), όπου $\mu$ είναι το τελευταίο ψηφίο του αριθμού μητρώου σας. Ελέγξτε με το χέρι αν τα παραπάνω διανύσματα είναι γραμμικά ανεξάρτητα ή όχι χωρίς την χρήση του κριτηρίου της ορίζουσας. Αποτελούν αυτά τα διανύσματα βάση του $R^{3}$; Κατόπιν φτιάξτε ένα πρόγραμμα που να αποφασίζει αν μία οποιαδήποτε τριάδα διανυσμάτων στο $R^{3}$ αποτελεί βάση του $R^{3}$ κάνοντας χρήση του κριτηρίου της ορίζουσας.

\begin{align*}
    \text{(3.1.1)}\systeme[ijk]{
        u=3i-4j+5k,v=2i-3j+k,w=i-j+$\mu$k
    }
\end{align*}

\newpage
\section{Απάντηση}

\chapter{Άσκηση 4}
\section{Εκφώνηση}

Θεωρήστε τα διανύσματα του (4.1.1). Δείξτε ότι αυτά τα διανύσματα αποτελούν βάση του $R^{3}$. Εφαρμόστε με το χέρι τον αλγόριθμο Gram-Smith για την ορθοκανονικοποίηση της βάσης. Φτιάξτε ένα πρόγραμμα που θα δέχεται σαν είσοδο τρία διανύματα σαν στήλες ενός $3x3$ πίνακα, θα ελέγχει αν αυτά τα διανύσματα αποτελούν βάση του $R^{3}$, και κατόπιν θα εφαρμόζει τον αλγόριθμο Gram-Smith και θα δίνει σαν έξοδο μία ορθοκανονική βάση.

\begin{align*}
    \text{(4.1.1)}\systeme[ijk]{
        u=i-2j+3k,v=2i-j+2k,w=i-2j+k
    }
\end{align*}


\newpage
\section{Απάντηση}

\end{document}