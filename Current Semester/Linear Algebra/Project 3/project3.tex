\documentclass[12pt, fleqn, leqno]{extreport}
\usepackage{perpage}
%\documentclass[12pt]{extreport}
\usepackage[margin=2cm,includeheadfoot,a4paper]{geometry}
\usepackage{fontspec}
%\usepackage[utf8x]{inputenc}
\usepackage[english,greek]{babel}
\usepackage{indentfirst}
\usepackage[dvipsnames]{xcolor}
\usepackage{listings}
\usepackage{titlesec}
\usepackage{amsmath, mathtools}
\usepackage{xifthen, xparse}
\usepackage{fancyhdr}
\usepackage{fancyvrb}
\usepackage[hyphens]{url}
\usepackage{hyperref}

\MakePerPage{footnote} 

%\setlength{\mathindent}{0pt}
\setlength{\headheight}{17pt}

%\renewcommand{\arraystretch}{1.5}

\titleformat{\chapter}[display]
  {\normalfont\bfseries}{}{0pt}{\Huge}

\setlength{\parskip}{0cm}
\setlength{\parindent}{1cm}

\setmainfont{[EBGaramond-Regular.ttf]}
\setmonofont{[FiraMono-Regular.otf]}

\hypersetup{
    colorlinks = true,
    linkcolor=black,
    filecolor=magenta,
    urlcolor=blue,
    pdftitle={Project 2}
}


%\definecolor{name}{model}{color-spec}

\lstdefinestyle{mystyle}{
    language=Octave,
    backgroundcolor=\color{white},   
    commentstyle=\color{teal},
    keywordstyle=\color{blue},
    numberstyle=\color{gray}\ttfamily,
    stringstyle=\color{orange},
    basicstyle=\ttfamily\footnotesize,
    breakatwhitespace=false,         
    breaklines=true,                 
    captionpos=b,                    
    keepspaces=true,                 
    numbers=left,                    
    numbersep=5pt,                  
    showspaces=false,                
    showstringspaces=false,
    showtabs=false,                  
    tabsize=2,
    frame=lines,
    framesep=0.1cm,
    rulecolor=\color{black},
    morestring=[b]"
}

\lstset{style=mystyle}

\pagenumbering{arabic}

\pagestyle{fancy}
\fancyhf{}
\fancyhead[R]{\rightmark}
\lhead{Project 2}
\chead{Αριθμητική Γραμμική Άλγεβρα}
\cfoot{\thepage}

\newcommand\rowop[1]{\scriptstyle\smash{\xrightarrow[\vphantom{#1}]{\mkern-4mu#1\mkern-4mu}}}

\DeclareDocumentCommand\converttorows
{>{\SplitList{,}}m}
{\ProcessList{#1}{\converttorow}}
\NewDocumentCommand{\converttorow}{m}
{\ifthenelse{\isempty{#1}}{}{\rowop{#1}}\\}

\DeclareDocumentCommand \rowops{m}
{\;
 \begin{matrix}
\converttorows {#1}
 \end{matrix}
 \; }


\begin{document}

\title{Αριθμητική Γραμμική Άλγεβρα\\Project 3}
\author{Μιχαήλ Ανάργυρος Ζαμάγιας\\TP5000}
\date{\today}
\maketitle
\newpage

\tableofcontents

\chapter{Άσκηση 1}

\section{Ερώτημα}
Φτιάξτε μια συνάρτηση σε Octave που να δέχεται έναν nxn πίνακα ως δεδομένα και να υπολογίζει το χαρακτηριστικό πολυώνυμο του πίνακα, τα ιδιοδιανύσματα και τις ιδιοτιμές του πίνακα, χρησιμοποιώντας τις έτοιμες εντολές του Octave. Η συνάρτηση θα πρέπει να παίρνει σαν είσοδο τον πίνακα και να δίνει σαν έξοδο ένα πίνακα με στήλες τα μοναδιαία ιδιοδιανύσματα, ένα οριζόντιο διάνυσμα με τις ιδιοτιμές στις ίδια σειρά των ιδιοδιανυσμάτων και το χαρακτηριστικό πολυώνυμο σε μορφή διανύσματος συντελεστών (αριθμητική μορφή πολυωνύμων στο Matlab και Octave). Η συνάρτηση αυτή δεν πρέπει να κάνει χρήση του συμβολικού πακέτου του Octave γιατί αυτό είναι πολύ πιο αργό από το αριθμητικό πακέτο. Δώστε τα αποτελέσματα της συνάρτησης σας στον πίνακα \eqref{eq:11}. Εδώ $\mu$ είναι το τελευταίο ψηφίο του αριθμού μητρώου σας. Επαληθεύονται οι σχέσεις του Vieta για αυτόν τον πίνακα;
\begin{equation}%eq11
    A = \begin{pmatrix}
        1 & 2 & 3   \\
        2 & 4 & 5   \\
        3 & 5 & \mu
    \end{pmatrix}\label{eq:11}
\end{equation}

\newpage
\section{Απάντηση}



\chapter{Άσκηση 2}
\section{Ερώτημα}

\newpage
\section{Απάντηση}

\chapter{Άσκηση 3}
\section{Ερώτημα}

\newpage
\section{Απάντηση}

\chapter{Άσκηση 4}
\section{Ερώτημα}

\newpage
\section{Απάντηση}

\chapter{Πηγές}
\section{Σύνδεσμοι}
\section{Video}
\section{Αρχεία}




\end{document}