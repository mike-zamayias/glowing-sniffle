% !TEX encoding = UTF-8 Unicode
\documentclass[12pt]{report}
\usepackage[margin=2cm,includeheadfoot,a4paper]{geometry}
\usepackage[utf8x]{inputenc}
\usepackage[english,greek]{babel}
\usepackage{indentfirst}
\usepackage{multicol}
\usepackage{ucs}
\usepackage{microtype}
\usepackage{fancyhdr}
\usepackage{epigraph}
\usepackage[autostyle,english=american]{csquotes}
\usepackage[dvipsnames]{xcolor}
\usepackage{titlesec}
\usepackage{listings}
\usepackage[hyphens]{url} 
\usepackage[unicode]{hyperref}

\setlength{\headheight}{17pt}

%\defaultfontfeatures{Ligatures=TeX}
%\MakeOuterQuote{"}

\setlength{\parskip}{0cm}
\setlength{\parindent}{1cm}

%\setmainfont{[EBGaramond-Regular-Regular.ttf]}
%\setmonofont{[FiraMono-Regular.ttf]}

\hypersetup{
    colorlinks = true,
    linkcolor=black,
    filecolor=magenta,      
    urlcolor=blue,
    pdftitle={Τεχνολογία Πολυμέσων}
}

\lstdefinestyle{mystyle}{
    language=Python,
    backgroundcolor=\color{white},   
    commentstyle=\color{teal},
    keywordstyle=\color{blue},
    numberstyle=\color{gray}\ttfamily,
    stringstyle=\color{orange},
    basicstyle=\ttfamily\footnotesize,
    breakatwhitespace=false,         
    breaklines=true,                 
    captionpos=b,                    
    keepspaces=true,                 
    numbers=left,                    
    numbersep=5pt,                  
    showspaces=false,                
    showstringspaces=false,
    showtabs=false,                  
    tabsize=2,
    frame=lines,
    framesep=0.1cm,
    rulecolor=\color{black},
    morestring=[b]"    
    }

\lstset{style=mystyle}


\pagestyle{fancy}
\fancyhf{}
\rhead{Τεχνολογία Πολυμέσων}
\lhead{}
\cfoot{\thepage}

\title{Τεχνολογία Πολυμέσων}
\author{Μιχαήλ Ανάργυρος Ζαμάγιας -- ΤΠ5000}
\date{\today}

\begin{document}

\maketitle

\tableofcontents
%\newpage

\chapter{Πολυμέσα, διαδραστικότητα, υπερμέσα, κείμενο}
\newpage
\section{Πολυμέσα}
\begin{multicols*}{2}
    \subsection{Ορισμός}
    Η χρήση πολλών μέσων για την παρουσίαση της πληροφορίας.
    \subsection{Είδη πολυμέσων}
    \begin{itemize}
        \item Κείμενο
        \item Εικόνα (κινητή, ακίνητη)
        \item Ήχος και γραφικά
        \item Σχεδιοκίνηση (\textlatin{animation})
        \item \textlatin{Video}
        \item Ανάδραση (π.χ. \textlatin{force feedback, joysticks})
    \end{itemize}
    \subsection{Εφαρμογές}
    \begin{itemize}
        \item Εκπαίδευση
        \item Επικοινωνία
        \item Ενημέρωση
        \item Ψυχαγωγία
    \end{itemize}
    \subsection{Παραδείγματα}
    \begin{itemize}
        \item Τηλεόραση
        \item Κινηματογράφος
        \item Εικονογραφημένα βιβλία
        \item Ηλεκτρονικά παιχνίδια
    \end{itemize}
    \subsection{Χαρακτηριστικά}
    \begin{itemize}
        \item Ψηφιακή αναπαράσταση της πληροφορίας
        \item Διαδραστικότητα (\textlatin{interactivity})
        \item Ανάγκη συμπίεσης
    \end{itemize}
    \subsection{Κατηγορίες}
    \subsubsection{Συνθετικά ή Συλληφθέντα\\(\textlatin{synthesized or captured})}
    \begin{itemize}
        \item \textbf{Στη σύνθεση, η πληροφορία δημιουργείται} εξολοκλήρου στον υπολογιστή με κατάλληλα λογισμικά, π.χ.:
              \begin{itemize}
                  \item πληκτρολόγηση κειμένου σε κειμενογράφο
                  \item δημιουργία εικόνας στο \textlatin{Paint} ή \textlatin{Photoshop}
              \end{itemize}
        \item \textbf{Στη σύλληψη, η πληροφορία εισάγεται} με κάποια μέθοδο ψηφιοποίησης, π.χ.:
              \begin{itemize}
                  \item σάρωση εικόνας ή κειμένου στο \textlatin{scanner}
                  \item εισαγωγή \textlatin{video} από αναλογική κάμερα με κάρτα \textlatin{video capture}
              \end{itemize}
    \end{itemize}
    \subsubsection{Διακριά ή Συνεχή\\(\textlatin{discrete or continuous})}
    \subsubsection{Τοπικές ή Δικτυακές εφαρμογές πολυμέσων}
\end{multicols*}
\newpage
\chapter{Ήχος, ψηφιοποίηση, συμπίεση}
\newpage
\newpage
\chapter{Εικόνα, χρωματικά μοντέλα, ψηφιοποίηση, συμπίεση}
\newpage
\newpage
\chapter{\textlatin{Video}, συμπίεση}
\newpage
\newpage

\end{document}