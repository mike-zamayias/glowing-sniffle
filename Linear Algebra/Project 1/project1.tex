\documentclass[a4paper,12pt]{article}
\usepackage{inconsolata}
\usepackage{listings}
\usepackage[margin=2cm]{geometry}
\usepackage[LGR, T1]{fontenc}
\usepackage[utf8]{inputenc}
\usepackage[greek, english]{babel}
\usepackage{alphabeta}
\usepackage{amsfonts, amsmath, amssymb}
\usepackage{fixltx2e}
\usepackage{subfig}
\usepackage{float}
\usepackage{xcolor}
\definecolor{codeblue}{RGB}{44,133,217}
\definecolor{codegray}{RGB}{138,150,150}
\definecolor{codepurple}{RGB}{0.58,0,0.82}
\definecolor{backcolour}{RGB}{255,255,255}
\lstdefinestyle{mystyle}{
    backgroundcolor=\color{backcolour},
    commentstyle=\color{codegray},
    keywordstyle=\color{codeblue},
    numberstyle=\color{codegray}\ttfamily\bfseries,
    stringstyle=\color{codepurple},
    basicstyle=\ttfamily\small\bfseries,
    breakatwhitespace=false,
    breaklines=true,
    captionpos=b,
    keepspaces=true,
    numbers=left,
    numbersep=5pt,
    showspaces=true,
    showstringspaces=false,
    showtabs=false,
    tabsize=2
}
\lstset{style=mystyle}
\begin{document}
\begin{titlepage}
    \begin{center}
        \vspace*{\fill}
        \huge{\textbf{Αριθμητική Γραμμική Άλγεβρα\\}}
        \vfill
        \huge{\textbf{Project 1}}
        \vspace*{\fill}
        \vfill
        \normalsize\textbf{Ζαμάγιας Μιχάλης\\}
        \small\textbf{Εαρινό Εξάμηνο, 2020\\}
        \vfill
    \end{center}
\end{titlepage}
\tableofcontents
\newpage\section{Άσκηση 1}
\subsection{Εκφώνηση}
Φτιάξτε ένα πρόγραμμα σε Octave που να δέχεται έναν ${2x2}$ συμμετρικό πίνακα ως δεδομένα και να
υπολογίζει το χαρακτηριστικό πολυώνυμο του πίνακα. Κάντε το να βρίσκει τις ιδιοτιμές του πίνακα
λύνοντας το χαρακτηριστικό πολυώνυμο και μετά, για κάθε ιδιοτιμή να βρίσκει ένα μοναδιαίο ιδιοδιάνυσμα
λύνοντας μία από τις εξισώσεις που δίνουν το ιδιοδιάνυσμα. Επαληθεύστε τα αποτελέσματα του προγράμματός
σας δίνοντας την έτοιμη εντολή διαγονοποίησης "eig" που έχει το Octave. Δώστε τα αποτελέσματα του
προγράμματός σας στον πίνακα
$
    A=\begin{pmatrix}
        1 & 2 \\
        2 & 1
    \end{pmatrix}
$
και
$
    A=\begin{pmatrix}
        1   & \mu \\
        \mu & 1
    \end{pmatrix}
$
. Εδώ ${\mu}$ είναι το τελευταίο ψηφίο του αριθμού μητρώου σας.
\subsection{Λύση}
\newpage\section{Άσκηση 2}
\subsection{Εκφώνηση}
Σκοπός της άσκησης είναι να δούμε την δράση ενός συμμετρικού πίνακα A πανω στα διανύσματα θέσης (σημεία)
του επιπέδου. Φτιάξτε ένα πογραμμα που θα δέχεται ένα διάνυσμα
$
    \begin{pmatrix}
        x \\
        y
    \end{pmatrix}
$
στο επίπεδο και θα το μετασχηματίζει στο διάνυσμα
$
    A
    \begin{pmatrix}
        x \\
        y
    \end{pmatrix}
$
. Για την γραφική απεικόνιση των δύο διανυσμάτων, ζωγραφίσετε μία μπλε γραμμή από την αρχή των αξόνων
στο σημείο
$
    \begin{pmatrix}
        x \\
        y
    \end{pmatrix}
$
, και μία κόκκινη γραμμή από την αρχή των αξόνων στο σημείο
$
    A
    \begin{pmatrix}
        x \\
        y
    \end{pmatrix}
$. Δώστε τα αποτελέσματα σας για
$
    A=\begin{pmatrix}
        1 & 2 \\
        2 & 1
    \end{pmatrix}
$
και για
$
    \begin{pmatrix}
        x \\
        y
    \end{pmatrix}=\begin{pmatrix}
        1 \\
        0
    \end{pmatrix}
$
, καθώς και για
$
    \begin{pmatrix}
        x \\
        y
    \end{pmatrix}=\begin{pmatrix}
        0 \\
        1
    \end{pmatrix}
$
. Ποια είναι η σχέση των μετασχηματισμένων διανυσμάτων με τον πίνακα Α;
\subsection{Λύση}
\newpage\section{Άσκηση 3}
\subsection{Εκφώνηση}
Σκοπός της άσκησης είναι να δούμε πως δρα ένας συμμετρικός πίνακας Α σε μία καμπύλη στο επίπεδο.
Για αυτόν τον σκοπό θα ζωγραφίσετε (με Octave) ένα μοναδιαίο κύκλο στο επίπεδο με μπλε χρώμα.
Για να επιτευχθεί εύκολα αυτό χρησιμοποιήστε την παραμετρική μορφή του μοναδιαίου κύκλου, $x=\cos{t}$,
$y=\sin{t}$. Μετά θα μετασχηματίσετε τον κύκλο σύμφωνα με τον πίνακα Α, μετασχηματίζοντας όλα τα σημεία
του με τον πίνακα Α. Εν γένη, από αυτόν τον μετασχηματισμό προκύπτει μια έλλειψη. Αυτήν την έλλειψη θα την
ζωγραφίσετε με κόκκινο χρώμα στο ίδιο σύστημα αξόνων με τον κύκλο. Δώστε τα αποτελέσματά σας για
$
    A=\begin{pmatrix}
        1 & 2 \\
        2 & 1
    \end{pmatrix}
$
και
$
    A=\begin{pmatrix}
        1   & \mu \\
        \mu & 1
    \end{pmatrix}
$
.
\subsection{Λύση}
\newpage\section{Άσκηση 4}
\subsection{Εκφώνηση}
Σκοπός της άσκησης είναι να διαβάσουμε τα ιδιοδιανύσματα από την δράση ενός συμμετρικού πίνακα Α στο
επίπεδο. Πάνω στο γράφημα με τον κύκλο και την έλλειψη της άσκησης 3, ζωγραφίστε και τα ιδιοδιανύσματα
του πίνακα Α ως ευθύγραμμα τμήματα που ξεκινούν από την αρχή τον αξόνων. Το μήκος κάθε ιδιοδιανύσματος
πάρτε το να είναι ίσο με την ιδιοτιμή που αντιστοιχεί με αυτό, κατ' απόλυτη τιμή. Τι παρατηρείτε; Δώστε
τα αποτελέσματά σας για
$
    A=\begin{pmatrix}
        1 & 2 \\
        2 & 1
    \end{pmatrix}
$
και
$
    A=\begin{pmatrix}
        1   & \mu \\
        \mu & 1
    \end{pmatrix}
$
.
\subsection{Λύση}
\newpage\section{Σημειώσεις}
Σε όλες τις παραπάνω ασκήσεις τα προγράμματαθα πρέπει να λειτουργούν για όλους τους συμμετρικούς πίνακες
που έχουν δύο διαφορετικές ιδιοτιμές. Τα όρια των αξόνων θα πρέπει να επιλεγούν κατάλληλα έτσι ώστε να
φαίνονται οι μετασηματισμοί.
\end{document}