\documentclass[a4paper,12pt]{article}
\usepackage{inconsolata}
\usepackage{listings}
\usepackage[margin=2cm]{geometry}
\usepackage[LGR, T1]{fontenc}
\usepackage[utf8]{inputenc}
\usepackage[greek,english]{babel}
\usepackage{alphabeta}
\usepackage{amsfonts, amsmath, amssymb}
\usepackage{fixltx2e}
\usepackage{subfig}
\usepackage{float}



\begin{document}

\begin{titlepage}
    \begin{center}
        \vspace*{\fill}
        \huge\textbf{Σημειώσεις Τεχνολογία Πολυμέσων\\}
        \vspace*{\fill}
        \vfill
        \normalsize\textbf{Ζαμάγιας Μιχάλης\\}
        \small\textbf{Ιανουάριος 2020\\}
        \vfill
    \end{center}
\end{titlepage}

\tableofcontents

\newpage\section{Ψηφιοποίηση}
\subsection{Τα 3 στάδια}
Η μετατροπή ενός αναλογικού σήματος σε ψηφιακό, δηλαδή η
μετατροπή μιας συνεχούς συνάρτησης σε μια σειρά διακριτών
τιμών, γίνεται ως εξής:
\begin{enumerate}
    \item Δειγματοληψία (sampling)
    \item Κβάντωση (quantization)
    \item Κωδικοποίηση (coding)
\end{enumerate}
\begin{itemize}
    \item Η παλμωδική διαμόρφωση (pulse-code modulation)
          αναφέρεται σε σήμα στο οποίο έχει γίνει δειγματοληψία,
          κβάντωση και κωδικοποίηση.
\end{itemize}

\newpage\section{Δειγματοληψία (sampling)}
\subsection{Καταγραφή Δειγμάτων}
Το σήμα καταγράφεται σε τακτά χρονικά διαστήματα, δημιουργώντας
μια σειρά δειγμάτων (samples) που προσεγγίζει την μορφή του
αναλογικού σήματος.
\subsection{Συχνότητα Δειγματολήψιας}
Τα περισσότερα σήματα είναι περιορισμένου εύρους ζώνης, δηλαδή
υπάρχει μια συχνότητα αποκοπής $f_{n}$ πάνω από την οποία η
ενέργεια του σήματος είναι πρακτικά μηδέν.
\subsection{Συχνότητα Nyquist}
Η ελάχιστη συχνότητα δειγματοληψίας (sampling frequency) $f_{s}$
είναι διπλάσια από την συχνότητα αποκοπής του σήματος $f_{n}$.

\newpage\section{Κβάντωση (quantization)}
\subsection{Ορισμός}
Η δημιουργία κατάλληλης κλίμακας διακριτών δειγμάτων για το
είδος των δεδομένων που έχουμε.
\subsection{Μέγεθος Δείγματος}
Το πλήθος των διαθέσιμων δειγμάτων για την αναπάρασταση του
αρχικού μεγέθους. \textit{Επηρεάζει}:
\begin{itemize}
    \item Το μέγεθος του σφάλματος κβάντωσης (quantization error).
    \item Τη μνήμη που απαιτείται για την αποθήκευση μιας
          ψηφιοποιημένης τιμής.
\end{itemize}
\subsection{Σφάλμα Κβάντωσης}
\begin{itemize}
    \item Είναι το σφάλμα στρογγυλοποίησης που εισάγει η διαδικασία κβάντωσης.
    \item Η συνολική αρμονική παραμόρφωση (Total Harmonic Distortion / THD) ενός
          σήματος ήχου μετρά την συνεισφορά των αρμονικών (πέρα της βασικής συχνότητας)
          στην έντασ του σήματος, και συνεπώς επηρεάζει την πιστότητα στην εγγραφή /
          αναπαραγωγή του ήχου:\\
          \begin{center}
              $THD = \frac{\sum harmonic powers}{fundemental frequency power}$
          \end{center}
\end{itemize}


\newpage\section{Κωδικοποίηση}

\subsection{Delta}
\begin{itemize}
    \item Καταγράφεται η διαφορά της τιμής κάθε δεδομένου μετά από το αρχικό δεδομένο.
    \item Είναι αποδοτική όταν οι τιμές μεταβάλλονται με ομαλό τρόπο ώστε η διαφορά μεταξύ
    2 διαδοχικών τιμών να είναι μικρή.
    \item Μειώνει το πλάτος του καταγραφόμενου σήματος.
    \item Μετατρέπει τα δεδομένα σε μορφή κατάλληλη για εφαρμογή Huffman ή RLE.
\end{itemize}
\subsection{Huffman}
\begin{itemize}
    \item Σκοπός είναι να αντικατασταθούν:
    \begin{itemize}
        \item οι πιο συχνές εμφανίσεις με μικρότερα σύμβολα λιγότερων bit
        \item οι λιγότερο συχνές εμφανίσεις με μεγαλύτερα σύμβολα περισσότερων bit
    \end{itemize}
    \item \textit{Αλγόριθμος}:
    \begin{enumerate}
        \item Εντοπίστε τα δύο λιγότερο πιθανά σύμβολα και συνδυάστε τα σε ένα νέο σύμβολο με 
        πιθανότητα εμφάνισης το άθροισμα των επιμέρους συμβόλων.
        \item Επαναλάβετε το 1ο βήμα έως ότου απομείνει ένα μόνο σύμβολο με πιθανότητα 1.
        \item Στο δέντρο που σχηματίζεται σημειώστε "0" για κάθε αριστερό κλαδί και "1" για κάθε
        δεξί.
        \item Ο κώδικας Huffman για κάθε αρχικό σύμβολο είναιι η ακολουθία (path) των "0" και "1"
        που οδηγεί σε αυτό, αρχίζοντας από την ρίζα με πιθανότητα 1.
    \end{enumerate}
\end{itemize}
\subsection{RLE (Run-Length-Encoding)}
Γίνεται αντικατάσταση της κάθε ακολουθίας ενός συγκεκριμένου
συμβόλου από 2 σύμβολα, ως εξής:
\begin{enumerate}
    \item Τον αριθμό εμφάνισης του συμβόλου στην συγκεκριμένη
          ακολουθία.
    \item Το ίδιο το σύμβολο.
\end{enumerate}
Η κωδικοποίηση μπορεί να εφαρμοστεί σε όσα σύμβολα θέλουμε.
\subsection{PCM (Pulse Code Modulation)}
\begin{itemize}
    \item Απλή και ευρεία χρήση.
    \item Χρησιμοποιεί γραμμική κβάντωση.
    \item Αποθηκεύει ένας προς ένα τα δείγματα σε ψηφιακή μορφή.
    \item Δεν εφαρμόζει συμπίεση, τα αρχεία είναι μεγάλα σε μέγεθος.
    \item Η ποιότητα αναπαραγωγής εξαρτάται μόνο από:
    \begin{itemize}
        \item την ποιότητα δειγματοληψίας,
        \item την διαδικασία κβάντωσης.
    \end{itemize}
\end{itemize}
\subsection{Delta Modulation}


\newpage\section{MPEG Audio Compression}
\begin{itemize}
    \item Διαθέτει 3 επίπεδα συμπίεσης, όπου κάθε επόμενο επίπεδο:
    \begin{itemize}
        \item Καταλαβαίνει τα προηγούμενα επίπεδα,
        \item Υλοποιεί περισσότερη πολυπλοκότητα του ψυχοακουστικού μοντέλου με αποτέλεσμα
        να εφαρμόζει:
        \begin{itemize}
            \item Καλύτερη συμπίεσης για την ίδια ποιότητα ήχου, ή
            \item Καλύτερη ποιότητα για τον ίδιο ρυθμό μετάδος.
        \end{itemize}
    \end{itemize}
    \item Το μεγαλύτερο τμήμα της πολυπλοκότητας βρίσκεται στη μεριά του endoder και όχι του
    decoder.
    \item Layers
    \begin{itemize}
        \item MPEG-1 Layer 1
        \begin{itemize}
            \item Εφαρμογή:
            \begin{enumerate}
                \item Στο σύστημα συμπίεσης της ψηφιακής κασέτας DAT της Philips
            \end{enumerate}
            \item Συμπίεση 4:1
            \item Ηχητική ποιότητα: μέτρια
            \item Ρυθμός μετάδοσης: 192 ή 256 kbps/κανάλι
        \end{itemize}
        \item MPEG-1 Layer 2
        \begin{itemize}
            \item Εφαρμογή:
            \begin{enumerate}
                \item Στο ραδιόφωνο DAB
                \item Στο σύστημα DVB της ψηφιακής δορυφορικής τηλεόρασης
                \item Στον ήχο CD και DVD
            \end{enumerate}
            \item Συμπίεση 6:1 έως 8:1
            \item Ηχητική ποιότητα: εφάμιλλη του CD
            \item Ρυθμός μετάδοσης: 96 ή 128 kbps/κανάλι
        \end{itemize}
        \item MPEG-1 Layer 3
        \begin{itemize}
            \item Εφαρμογή:
            \begin{enumerate}
                \item Μεταφορά και φόρτωση μέσω δικτύου
                \item Ανάκληση και αναπαραγωγή από σκληρό δίσκο
            \end{enumerate}
            \item Συμπίεση 12:1
            \item Ηχητική ποιότητα: εφάμιλλη του CD
            \item Ρυθμός μετάδοσης: 64 kbps/κανάλι
        \end{itemize}
    \end{itemize}
    \item Βασικός Αλγόριθμος
    \begin{itemize}
        \item Encoder
        \begin{enumerate}
            \item Audio (PCM) input
            \item Time to frequency transformation, Psychoacoustic modeling
            \item Bit allocation, quantizing and coding
            \item Bitstream formating
            \item Endocded bitstream
        \end{enumerate}
        \item Decoder
        \begin{enumerate}
            \item Endocded bitstream
            \item Bitstream unpacking
            \item Frequency sample reconstruction
            \item Frequency to time transformation
            \item Decoded PCM audio
        \end{enumerate}
    \end{itemize}
    \item Audio Layers
    \begin{itemize}
        \item Layers 1 and 2
        \begin{enumerate}
            \item PCM audio signal
            \item Filter bank: 32 subbands, Psychoacoustic model on top of 1024-point FFT
            \item Linear quantizer, Side-information coding on top of Linear quantizer
            \item Bitstream formating
            \item Coded audio signal
        \end{enumerate}
        \item Audio Layer 3
        \begin{enumerate}
            \item PCM audio signal
            \item Filter bank: 32 subbands, Psychoacoustic model on top of 1024-point FFT
            \item M-DCT
            \item Nonuniform quantization
            \item Side-information coding and Huffman coding
            \item Bitstream formating
            \item Coded audio signal
        \end{enumerate}
    \end{itemize}
    \item MP3
    \begin{enumerate}
        \item Βήμα 1
        \begin{enumerate}
            \item Το ψηφιοποιημένο σήμα σε μορφή PCM περνά από μια τράπεζα φίλτρων που του
            διαιρεί σε 32 μπάντες συχντήτων ίσου πλάτους.
            \item Layer 1
            \begin{itemize}
                \item Κάθε φίλτρο ομαδοποιεί 1 ομάδα των 12 δειγμάτων ανά ζώνη
                \item Για τις 32 ζώνες έχουμε 1*12*32=384 δείγματα
                \item Το μέγεθος ενός δείγματος είναι 15 bits
            \end{itemize}
            \item Layers 2 and 3
            \begin{itemize}
                \item Κάθε φίλτρο ομαδοποιεί 3 ομάδες των 12 δειγμάτων ανά ζώνη
                \item Για τις 32 ζώνες έχουμε 3*12*32=1152 δείγματα
                \item Το μέγεθος ενός δείγματος είναι 16 bits
            \end{itemize}
            \item Τα φίλτρα και τα αντίστροφά τους είναι lossy, όμως το σφάλμα που εισάγεται
            είναι πολύ μικρό και μη αντιλυπτό.
        \end{enumerate}
        \item Βήμα 2
        \begin{itemize}
            \item Το φάσμα συχνοτήτων υποδιαιρείται περαιτέρω για καλύτερη φασματική
            διακριτότητα
            \item Εφαρμόζεται τροποποιημένος διακριτός μετασχηματισμός συνημτόνου (MDCT) για να
            αντισταθμιστούν σφάλματα που εισάγουν τα τετράγωνα φίλτρα της τράπεζας φίλτρων:
            \begin{center}
                $X(m)=\sum_{k=0}^{n-1} {f(k)x(k)\cos[{\frac{\pi}{2n}(2k+1+\frac{n}{2})(2m+1)}]}$, $m=0...\frac{n}{2}-1$
            \end{center}
        \end{itemize}
        \item Βήμα 3
        \item Βήμα 4
    \end{enumerate}
\end{itemize}

\newpage\section{Μην ξεχάσεις}
\subsection{Στάδια συμπίεσης JMPEG}
\subsection{Αντικειμενικά και υποκειμενικά χαρακτηριστικά ήχου}
\subsection{Διαπλασιακή συμπίεσης}
\subsection{Τα βήματα της συμπίεσης κατά MPEG}
\subsection{Καμπύλες ίσης ακουστικότητας}
\begin{itemize}
    \item Δείχνουν την σχέση έντασης (P) - συχνότητας (f) που ο μέσος ακροατής εκλαμβάνει
    ως ίδιας ακουστικότητας.
    \item Είναι κανονικοιποιημένες ώστε η ένταση να είναι ίδια με την ακουστικότητα σε
    συχνότητα 1 kHz.
\end{itemize}

\end{document}