\documentclass[a4paper,10pt]{report}
\usepackage[margin=2cm]{geometry}
\usepackage{fontspec}
\usepackage[utf8x]{inputenc}
\usepackage[english,greek]{babel}
\usepackage{fancyhdr}
\usepackage{epigraph}
\usepackage{xcolor}
\usepackage{listings}
\usepackage[hyphens]{url} 
\usepackage{hyperref}

\setmainfont{[FiraSans-Book.otf]}
\setmonofont{[FiraMono-Regular.otf]}

\hypersetup{
    colorlinks = true,
    linkcolor=black,
    filecolor=magenta,      
    urlcolor=blue,
    pdftitle={Προγραμματισμός, σε Python},
    pdfpagemode=FullScreen
}

\lstdefinestyle{mystyle}{
    basicstyle=\ttfamily\normalsize,
    backgroundcolor=\color{lightgray},
    xleftmargin=2cm,
    breakatwhitespace=false,         
    breaklines=true,                 
    captionpos=b,                    
    keepspaces=true,                 
    numbers=left,                    
    numbersep=5pt,                  
    showspaces=false,                
    showstringspaces=false,
    showtabs=false,                  
    tabsize=4
}
\lstset{style=mystyle}

\renewcommand {\epigraphflush}{center}

\pagestyle{fancy}
\fancyhf{}
\rhead{
    Προγραμματισμός, σε Python
    }
\lhead{
    Διδακτική της Πληροφορικής
}
        
\title{Προγραμματισμός, σε Python}
\author{Μιχαήλ Ανάργυρος Ζαμάγιας}
\date{19/5/2020}

\begin{document}

\maketitle

\tableofcontents

\chapter{Εισαγωγή}

\section{Πρόλογος}
\paragraph{
    Άρχισα να γράφω Python στην πρώτη λυκείου. Πρέπει να ήταν ένα χόμπι μου
    τότε, δεν θυμάμαι όμως τους ακριβείς λόγους που το ξεκίνησα. Σίγουρα
    πάντως μου είχε κεντρήσει το ενδιαφέρον. Ίσως επειδή είχα χάσει το
    ενδιαφέρον για το hardware ενώ παράλληλα να κέρδιζε το ενδιαφέρον
    μου η διαδικασία του να κάνω το hardware χρήσιμο για τον χρήστη, εμένα.
}
\paragraph{
    Κάποια στιγμή, μέχρι την δευτέρα λυκείου, είχα καταφέρει να γράψω ένα
    bot που έτρεχε στο terminal του υπολογιστή μου. Μπορούσε να απαντήσει σε
    όλες τις ερωτήσεις μου, δεδομένου ότι τις καταλάβαινε φυσικά. Δεν μπορούσε
    όμως να κρατήσει διάλογο. Άλλωστε ήταν ένα χαζό bot, όχι μια παντογνώστρια
    Γενική Τεχνιτή Νοημοσύνη. Και ήταν όντως ένα χαζό πρόγραμμα, όπως θα δείτε
    στο επόμενο κεφάλαιο.
}
\paragraph{
    Θυμάμαι ακόμα, ότι είχα γράψει ένα πρόγραμμα "μετερεολόγο" για μία σχολική
    εκδρομή, της δευτέρας ή τρίτης λυκείου. Ήταν ένα σχετικά μικρό πρόγραμμα,
    το οποίο έπερνε τα καιρικά δεδομένα από μια βάση δεδομένων και τα
    εμφάνιζε σε μια ευνόητη μορφή για τον άνθρωπο. Όμως, οι προβλέψεις δεν
    ήταν ακριβείς, συγκριτικά μέ άλλες υπηρεσίες, και δεν έψαξα άλλες
    υπηρεσίες δεδομένων, το άφησα.
}
\paragraph{
    Δεν θέλω να φανώ αλλαζόνας, ίσα ίσα. Θέλω να σας δείξω ότι μπρορείτε να
    χτίσετε αρκέτα και διαφορετικά πράγματα, έχοντας βάση το hardware φυσικά.
    Δεν θα με αποκαλούσα προγραμματιστή τότε, αντίθετα, απλά ήθελα να κάνω
    τον υπολογστή μου χρήσιμο σε εμένα και για τους δικούς μου λόγους. Οπότε,
    ξεκίνησα να μαθαίνω Python μόνος μου. Μια απλή και εύκολη γλώσσα
    προγραμματισμού, όμως με πολλές δυνατότες και με ευρεία χρήση, με σκοπό
    να καταφέρω τον στόχο μου.
}
\paragraph{
    Τώρα, μερικά χρόνια μετά, θα ήθελα να σας εξοικειώσω με τα βασικά αυτής
    της γλώσσας. Ακόμα περισσότερο, θα ήθελα να σας μεταδώσω το ίδιο
    ενδιαφέρον μου και σε εσάς, επειδή πιστεύω ότι μπορείτε να κάνετε
    ευκολότερη την ζωή σας και να γλυτώσετε πολύ χρόνο από εργασίες αγγαρίες.
    Καθώς έτσι θα είστε σε θέση να δημιουργείτε τα δικά σας εργαλεία για τις
    δικές ανάγκες σε έναν υπολογιστή. Τέλος, θα πάρει καιρό μέχρι να κάνετε κάτι
    εντυπωσιακό, αλλά όχι πολύ, οπότε μην απογοητευτείτε.
}
\section{Γιατί να ασχοληθώ με τον προγραμματισμό;}
\begin{center}
    \epigraph{
        "Codes are a puzzle. A game, just like any other game."
    }{— Alarn Turing}
\end{center}
\paragraph{
    Ο προγραμματισμός, ως γνώση, δίνει σε κάποιον την δεξιότητα να
    μετατρέπει ένα πρόβλημα σε κάτι που μπορεί να κατανοήσει και να λύσει,
    τις περισσότερες φορές, ένας υπολογιστής. Μάλιστα, θα υποστήριζα ότι
    είναι μια βασική δεξιότητα για κάποιον άνθρωπο στην εποχή μας,
    εάν εκείνος χρησιμοποιεί υπολογιστές.
}
\section{Γιατί να μάθω Python;}
\paragraph{
    Η Python είναι μια δερμηνευμένη, γενικού σκοπού και υψηλού επιπέδου γλώσσα
    προγραμματισμού. Δηλαδή:
}
\begin{itemize}
    \item μέρη προγράμματος μπορούν να εκτελστούν εκτός του κυρίου
          προγράμματος, μέσα από τον διερμηνέα,
    \item βρίσκεται πίσω από πολλές εφαρμογές και προσφέρει λύση σε πολλά
          προβλήματα, και
    \item το συντακτικό της είναι απλούστερο, ευκολότερο στην κατανόηση σε
          σύγκριση με άλλες γλώσσες προγραμματισμού.
\end{itemize}
\section{Ποιος είναι ο σκοπός αυτού του εγχειριδίου;}
\paragraph{
    Στόχος μου μέσα από αυτό το εγχειρίδιο είναι η εξοικείωση του αναγνώστη
    με την Python και η εισαγωγή του στον τρόπο σκέψης ενός προγραμματιστή.
    Το εγχειρίδιο χωρίζεται σε δύο μέρη. Στο πρώτο μέρος ο αναγνώστης μαθαίνει
    για τα βασικά της γλώσσας, βασικές έννοιες προγραμματισμού, είδη δεδομένων
    και τους τρόπους χρήσης τους. Στο δεύτερο μέρος, ο αναγνώστης θα εφαρμόσει
    πρακτικά όσα έμαθε στο πρώτο μέρος σε μικρά προγράμματα.
}

\chapter{Τα βασικά}
\paragraph{
    Τα βασικά είναι βαρετά. Επιτρέψτε μου να εξηγήσω περισσότερο. Λόγω της
    φύσης της γλώσσας δεν χρειάζεστε να παραμείνετε στα βασικά. Είναι αρκετά
    εύκολο, με μια καλή κατανόηση, των βασικών να χτίσετε σχεδόν οτιδήποτε
    φανταστείτε. Τα βασικά εργαλεία σε αυτήν την περίπτωση είναι οι προτάσεις
    υπό συνθήκη, οι επαναλήψεις, οι συναρτήσεις και τα δομοστοιχεία της γλώσσας.
}
\section{Ξεκινώντας}
\paragraph{
    Η Python δεν έρχεται προεγκατεστημένη με τα Windows και συνήθως στις
    Linux διανομές η προεγκατεστημένη έκδοση είναι η 2.7 η οποία πλέον έχει
    σταματήσει να λαμβάνει ενημερώσεις και υποστήριξη. Άρα, χρειάζεστε μια
    νεότερη έκδοση της Python. Τέλος, τα προγράμματα γραμμένα σε Python
    έχουν κατάληξη \colorbox{lightgray}{\lstinline{.py}}.
}
\subsection{Εγκατάσταση Python}
\paragraph{
    Η Python είναι μια cross-platform γλώσσα προγραμματισμού, με υποστήριξη
    σε Windows, Linux και macOS και με μερικές αλλαγές ανά την πλατφόρμα.
    Σε αυτό το εγχειρίδιο χρησιμοποιούνται οι
    \colorbox{lightgray}{\lstinline{Python 3.7.6}} και
    \colorbox{lightgray}{\lstinline{conda 4.8.3}}
    στην διανομή Linux \colorbox{lightgray}{\lstinline{Pop!_OS}}.
}
\paragraph{
    Σας προτείνω να εγκαταστήσετε την Anaconda, η οποία είναι μια δωρεάν
    και ανοιχτού κώδικα διανομή της Python. Η διαδικασία εγκατάστασης είναι
    εύκολη και προσφέρει ένα μεγάλο σύνολο εργαλείων σε ένα σημείο, κι ως
    αποτέλεσμα κάνει μελλοντικούς πειραματισμούς σας πιο εύκολους.
}
\subsubsection{Σε Windows}
Επισκεφθείτε την \href{https://tinyurl.com/yc39u67t}{επίσημη σελίδα}
και κατεβάστε το αρχείο εγκατάστασης για Windows με την τελευταία έκδοση της
Python. Ξεκινήστε κι ολοκληρώστε την εγκατάσταση, χωρίς αλλάξετε τις
προεπιλεγμένες ρυθμίσεις.
\subsubsection{Σε Linux}
Επισκεφθείτε την \href{https://tinyurl.com/yc39u67t}{επίσημη σελίδα}
και κατεβάστε το αρχείο εγκατάστασης για Linux με την τελευταία έκδοση της
Python. Ανοίξτε το terminal στην τοποθεσία που έγινε η λήψη του αρχείου
εγκατάστασης κι εκτελέστε την εντολή
\colorbox{lightgray}{\lstinline{bash ./"Anaconda"}}, όπου
\colorbox{lightgray}{\lstinline{"Anaconda"}} το όνομα αρχείου εγκατάστασης.
Η εγκατάσταση θα ξεκινήσει στο terminal, και προχωράτε την πρόοδο της
εγκατάστασης πατώντας \colorbox{lightgray}{\lstinline{Enter}}. Στο σημείο
\begin{center}
    \colorbox{lightgray}{\lstinline{
            Do you wish the installer to initialize
            Anaconda3 by running conda init?}}
\end{center}
απαντήστε \colorbox{lightgray}{\lstinline{yes}}, έτσι ώστε να έχετε πιο
εύκολη πρόσβαση στην γλώσσα την επόμενη φορα που θα χρησιμοποιείσετε την
γλώσσα.
\subsubsection{Επιβεβαίωση εγκατάστασης}
Σε Windows, αναζητήστε το \colorbox{lightgray}{\lstinline{Anaconda Prompt}}
στο μενού έναρξης ή σε Linux, ανοίξτε το terminal.
Εάν η εγκατάσταση ολοκληρώθηκε με επιτυχία, η εκτέλεση των δυο επόμενων
εντολών, \colorbox{lightgray}{\lstinline{python --version}} και
\colorbox{lightgray}{\lstinline{conda --version}}, θα έχει ένα παρόμοιο
αποτέλεσμα: \colorbox{lightgray}{\lstinline{Python 3.7.6}} και
\colorbox{lightgray}{\lstinline{conda 4.8.3}}, αντίστοιχα.
\subsection{Εγκατάσταση επεξεργαστή κειμένου}
\paragraph{
    Μπορούμε να γράψουμε Python σε οποιονδήποτε επεξεργαστή κειμένου, ακόμα και
    στο Σημειωματάριο που έρχεται με τον υπολογιστή μας. Υπάρχουν όμως άλλα
    προγράμματα τα οποία είναι φτιαγμένα για προγραμματισμό, τα οποία φέρουν
    χαρακτηριστικά που θα κάνουν την ζωή μας πιο εύκολη στην συνέχεια. Μερικά
    τέτοια δωρεάν και ανοιχτού κώδικα προγράμματα είναι τα εξής:
    Sublime Text Editor, VSCodium Text Editor και Atom Text Editor. Παρακάτω
    θα δείτε πώς να εγκαταστήσετε τον VSCodium.
}
\subsubsection{Σε Windows}
\subsubsection{Σε Linux}
\section{Η γλώσσα}
\subsection{Διερμηνευτής}
\subsection{Εκτέλεση προγραμμάτων}
\subsection{Εύρεση κι επίλυση σφαλμάτων}
\subsection{Σχόλια}
\subsection{Μεταβλητές}
\subsection{Τύποι δεδομένων}
\subsection{Συμβολοσειρές}
\subsubsection{Μορφοποίηση}
\subsubsection{Μέθοδοι}
\subsection{Λίστες}
\subsection{Tuples}
\subsection{Σύνολα}
\subsection{Λεξικά}
\subsection{Προτάσεις υπό συνθήκη}
\subsection{Βρόγχοι}
\subsection{Δομοστοιχεία}
\subsection{Δουλεύοντας με αρχεία}
\subsection{Δουλεύοντας με JSON αρχεία}

\chapter{Προγράμματα}

\chapter{Βιβλιογραφία}
\epigraph{\href{https://tinyurl.com/ycnad9ch}
    {Total 0 knowledge, entirely parallel with programming of any kind.
        Heard Python is simple, why would I want to learn it ?
    }
}{u/Azsras\_Zuralix on r/learnpython}
\section{Βιβλία}
\begin{itemize}
    \item \href{https://tinyurl.com/y7l2a48c}{Python Crash Course, 2nd
              Edition — by Eric Matthes}
\end{itemize}
\section{Video}
\begin{itemize}
    \item \href{https://tinyurl.com/ya8wk4xm}{Python Crash Course}
\end{itemize}
\section{Σύνδεσμοι}
\begin{itemize}
    \item \href{https://tinyurl.com/yyzfa2bg}{WordReference Dictionary}
    \item\href{https://tinyurl.com/o5vxal7}{Λεξικό της κοινής νεοελληνικής}
    \item \href{https://tinyurl.com/y9q2elk4}{Βιβλιογραφία, Wikipedia}
    \item \href{https://tinyurl.com/y9g9nkh2}{Python, Wikipedia}
    \item \href{https://tinyurl.com/ycy6jsw5}{Anaconda (Python distribution),
              Wikipedia}
    \item \href{https://tinyurl.com/y7rogsec}{Anaconda Individiual Edition,
              Anaconda | The World's Most Popular Data Science Platform}
    \item \href{https://tinyurl.com/ogoqf2p}{Conditional statements, Wikipedia}
    \item \href{https://tinyurl.com/y8y59y44}{Python Cheatsheet}
    \item \href{https://tinyurl.com/y54gclet}{VSCodium is a community-driven,
              freely-licensed binary distribution of Microsoft’s editor VSCode}
\end{itemize}
\section{Χρήσιμα αρχεία}
\begin{itemize}
    \item \href{https://tinyurl.com/y9l8o5n6}{Βιβλιογραφική ανασκόπηση,
              Δημοκρίτειο Πανεπιστήμιο Θράκης}
    \item \href{https://tinyurl.com/yaaswz5p}{Εισαγωγή στη LaTeX για φοιτητές.
              (An Introduction to Latex in Greek)}
    \item \href{https://tinyurl.com/nqbrvss}{Python Cheat Sheet}
\end{itemize}

\end{document}