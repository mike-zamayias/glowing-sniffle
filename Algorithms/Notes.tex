\documentclass[a4paper,12pt]{article}
\usepackage{inconsolata}
\usepackage{listings}
\usepackage[margin=2cm]{geometry}
\usepackage[LGR, T1]{fontenc}
\usepackage[utf8]{inputenc}
\usepackage[greek,english]{babel}
\usepackage{alphabeta}
\usepackage{amsfonts, amsmath, amssymb}
\usepackage{fixltx2e}
\usepackage{subfig}
\usepackage{float}
\usepackage{fancyvrb}

\lstdefinestyle{mystyle}{
    basicstyle=\ttfamily\footnotesize,
    breakatwhitespace=false,         
    breaklines=true,                 
    captionpos=b,                    
    keepspaces=true,                 
    numbers=left,                    
    numbersep=5pt,                  
    showspaces=false,                
    showstringspaces=false,
    showtabs=false,                  
    tabsize=4
}
\lstset{style=mystyle}


\begin{document}

\begin{titlepage}
    \begin{center}
        \vspace*{\fill}
        \huge\textbf{Ανάλυση Αλγορίθμων\\}
        \vspace*{\fill}
        \vfill
        \normalsize\textbf{Ζαμάγιας Μιχάλης\\}
        \small\textbf{Φεβρουάριος 2020\\}
        \vfill
    \end{center}
\end{titlepage}

\tableofcontents


\newpage
\section{Ανάλυση}
\begin{itemize}
    \item Ρυθμοί αύξησης συναρτήσεων, ταξινομημένα από γρήγορο σε αργό:
    \begin{enumerate}
        \item Σταθεροί όροι \\
        π.χ.:    $64$, $1000$
        \item Λογαριθμικές συναρτήσεις \\
        π.χ.:    $\log_8n$, $\log_2n$
        \item Γραμμικές συναρτήσεις \\
        π.χ.:    $4n$, $100n$
        \item Γραμμοαριθμικές συναρτήσεις \\
        π.χ.:    $n\log_8n$, $n\log_2n$
        \item Πολυωνυμικές συναρτήσεις \\
        π.χ.:    $8n^2$, $6n^3$
        \item Εκθετικές συναρτήσεις \\
        π.χ.:    $8^2n$, $6^3n$
    \end{enumerate}
    \item Ποια είναι η ασυμπτωτική σχέση μεταξύ των συναρτήσεων $n^k$ και $c^n$; Υποθέστε
    ότι $k >= 1$ και $c > 1$ σταθερές. Επιλέξτε τις σωστές απαντήσεις.
    \begin{enumerate}
        \item $n^k$ is $\mathcal{O}(c^n)$ \textbf{Σωστή}
        \item $n^k$ is $\Omega(c^n)$ \textbf{Λάθος}
        \item $n^k$ is $\Theta(c^n)$ \textbf{Λάθος}
    \end{enumerate}        
    \item Ποια είναι η ασυμπτωτική σχέση μεταξύ των συναρτήσεων $log_2n$ και $log_8n$; 
    Επιλέξτε τις σωστές απαντήσεις.
    \begin{enumerate}
        \item $log_2n$ is $\mathcal{O}(log_8n)$ \textbf{Σωστή}
        \item $log_2n$ is $\Omega(log_8n)$ \textbf{Σωστή}
        \item $log_2n$ is $\Theta(log_8n)$ \textbf{Σωστή}
    \end{enumerate}            
\end{itemize}

\newpage
\section{Αλγόριθμοι}
\subsection{Αναζήτησης}
\subsubsection{Δυαδική \textbf{(Binary)} - Επαναλυπτικός Αλγόριθμος}
\begin{lstlisting}[language=c]
    int iterativeBinarySearch(int pin[], int low, int high, int num)
    {
        int mid;
        while (low <= high)
        {
            mid = (low + high) / 2;
            if (pin[mid] == num)
            {
                return mid;
            }
            if (pin[mid] < num)
            {
                low = mid + 1;
            }
            else
            {
                high = mid - 1;
            }
        }
        return -1;
    }
\end{lstlisting}

\subsubsection{Δυαδική \textbf{(Binary)} - Αναδρομικός Αλγόριθμος}
\begin{lstlisting}[language=c]
    int recursiveBinarySearch(int pin[], int low, int high, int num)
    {
        int mid;
        if (low <= high)
        {
            mid = (low + high) / 2;
            if (pin[mid] == num)
            {
                return mid;
            }
            if (pin[mid] < num)
            {
                return recursiveBinarySearch(pin, mid+1, high, num)
            }
            else
            {
                return recursiveBinarySearch(pin, low, mid-1, num)
            }
        }
        return -1;
    }
\end{lstlisting}

\newpage
\subsection{Ταξινόμησης}
\subsubsection{Επιλογή \textbf{(Selection)} - Επαναλυπτικός Αλγόριθμος}
\begin{lstlisting}[language=c]
    void selectionSort(int pin[], int size)
    {
        int i, j, minPos, temp;
        for (i = 0; i <= size; j++)
        {
            minPos = i;
            for (j = i + 1; j <= size; j++)
            {
                if (pin[minPos] > pin[j])
                {
                    minPos = j;
                }
            }
            if (minPos != i)
            {
                temp = pin[i];
                pin[i] = pin[minPos];
                pin[minPos] = temp;
            }
        }
    }
\end{lstlisting}

\subsubsection{Εισαγωγή \textbf{(Insertion)} - Επαναλυπτικός Αλγόριθμος}
\begin{lstlisting}[language=c]
    void insertionSort(int pin[], int size)
    {
        int i, j, value;
        for (i = 1; i <= size; i++)
        {
            value = pin[i];
            j = i - 1;
            while (j >= 0 && pin[j] > value)
            {
                pin[j+1] = pin[j];
                j--;
            }
            pin[j+1] = value;
        }
    }
\end{lstlisting}

\newpage
\subsubsection{Αντιμετάθεση \textbf{(Bubble)} - Επαναλυπτικός Αλγόριθμος}
\begin{lstlisting}[language=c]
    void bubbleSort(int pin[], int size)
    {
        int i, j, temp;
        for (i = 1; i <= size; i++)
        {
            for (j = size; j >= i; j--)
            {
                if (pin[j] < pin[j-1])
                {
                    temp = pin[j];
                    pin[j] = pin[j-1];
                    pin[j-1] = temp;
                }
            }
        }
    }
\end{lstlisting}

\subsubsection{Συγχώνευση \textbf{(Merge)} - Αναδρομικός Αλγόριθμος}
\begin{lstlisting}[language=c]
    void mergeSort(int pin[], int low, int high)
    {
        int mid;
        if (low < high)
        {
            mid = (low + high) / 2;
            mergeSort(pin, low, mid);
            mergeSort(pin, mid+1, high);
            merge(pin, low, mid, mid+1, high);
        }
    }
\end{lstlisting}

\newpage
\subsubsection{Ταχεία Ταξινόμηση \textbf{(QuickSort)} - Αναδρομικός Αλγόριθμος}
\begin{lstlisting}[language=c]
    void quickSort(int pin[], int low, int high)
    {
        int pivot;
        if (low < high)
        {
            pivot = partition(pin, low, high);
            quickSort(pin, low, pivot-1);
            quickSort(pin, pivot+1, high);
        }
    }

    int partition(int pin[], int low, int high)
    {
        int pivot, smallerPos, i, tmp;
        
        pivot = pin[high];
        smallerPos = low - 1;
        
        for (i = low; i <= high - 1; i++)
        {
            if (pin[i] <= pivot)
            {
                smallerPos++;
                tmp = pin[smallerPos];
                pin[smallerPos] = pin[i];
                pin[i] = tmp;
            }
        }        

        tmp = pin[smallerPos + 1];
        pin[smallerPos + 1] = pin[high];
        pin[high] = tmp;
        
        return (smallerPos + 1);
    }
\end{lstlisting}

\newpage
\section{Γράφοι}
\subsection{Διάσχιση}
\subsubsection{BFS}
\subsubsection{DFS}
\subsection{Τοπολογική Αναζήτηση}
\subsection{Ελάχιστα Συνδετικά Δέντρα (MST)}
\subsubsection{Prim}
\subsubsection{Kruskal}


\end{document}