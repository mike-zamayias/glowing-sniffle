\documentclass[12pt]{extarticle}
% \documentclass[15pt, fleqn, leqno]{extarticle}
% \documentclass[12pt]{extreport}
\usepackage[margin=2cm,includeheadfoot,a4paper]{geometry}
\usepackage[english, greek]{babel}
\usepackage[normalize-symbols,keep-semicolon]{alphabeta}
% \usepackage{fontspec}
\usepackage{indentfirst}
\usepackage[dvipsnames]{xcolor}
\usepackage{titlesec}
\usepackage{multicol}
\usepackage{amsmath, mathtools}
\usepackage{fancyhdr}
\usepackage{fancyvrb}
\usepackage{perpage}
\usepackage[hyphens]{url}
% \usepackage{hyperref}

\MakePerPage{footnote} 

%\setlength{\mathindent}{0pt}
% \setlength{\headheight}{16pt}

%\renewcommand{\arraystretch}{1.5}

\titleformat{\chapter}[display]
  {\normalfont\bfseries}{}{0pt}{\Huge}

\setlength{\parskip}{0cm}
\setlength{\parindent}{1cm}

\pagenumbering{arabic}

\pagestyle{fancy}
\fancyhf{}
\fancyhead[R]{\rightmark}
\lhead{Εφαρμοσμένα Μαθηματικά}
% \chead{}
% \rhead{}
\cfoot{\thepage}

% \setmainfont{EB Garamond}

\begin{document}

\tableofcontents
\newpage
\chapter{}{Διάλεξη Πρώτη}
\section{Διαφορική Εξίσωση}
\begin{equation*}
    \begin{aligned}
        \alpha y''(x) + \beta y' (x)+ \gamma y = 0
    \end{aligned}
\end{equation*}
Η συνήθης Δ.Ε. είναι γραμμική, δεύτερης τάξης, ομογενής, με
\begin{math}
    α,β,γ
\end{math}
σταθερούς συντελεστές.
\section{Χαρακτηριστική Eξίσωση}
\begin{equation*}
    \begin{aligned}
        \alpha r^2 + \beta r + \gamma = 0
    \end{aligned}
\end{equation*}
Για την γενική λύση της Χ.Ε. διακρίνονται οι παρακάτω περιπτώσεις:
\begin{enumerate}
    \item \begin{math}
              \Delta   > 0 :
          \end{math}
          \begin{equation*}
              \begin{aligned}
                  r_1, r_2 & \in \Re                       \\
                  y(x)     & = c_1 e^{r_1x} + c_2 e^{r_2x} \\
                  y_1(x)   & = c_1 e^{r_1x}                \\
                  y_2(x)   & = c_2 e^{r_2x}                \\
                  c_1, c_2 & \text{ σταθερές}
              \end{aligned}
          \end{equation*}
    \item \begin{math}
              \Delta   = 0 :
          \end{math}
          \begin{equation*}
              \begin{aligned}
                  r        & \in \Re                     \\
                  y(x)     & = c_1 e^{rx} + c_2 x e^{rx} \\
                  y_1(x)   & = c_1 e^{rx}                \\
                  y_2(x)   & = c_2 x e^{rx}              \\
                  c_1, c_2 & \text{ σταθερές}
              \end{aligned}
          \end{equation*}
    \item \begin{math}
              \Delta   < 0 :
          \end{math}
          \begin{equation*}
              \begin{aligned}
                  r_1                      & = A + Bi                                    \\
                  r_2                      & = A - Bi                                    \\
                  A        = \frac{-β}{2α} & \text{, } B = \frac{\sqrt{-Δ}}{2α}          \\
                  y(x)                     & = c_1 e^{Ax} \sin{Bx} + c_2 e^{Ax} \cos{Bx} \\
                  y_3(x)                   & = c_1 e^{Ax} \sin{Bx}                       \\
                  y_4(x)                   & = c_2 e^{Ax} \cos{Bx}                       \\
                  c_1, c_2                 & \text{ σταθερές}
              \end{aligned}
          \end{equation*}
\end{enumerate}
\subsection{Απόδειξη περίπτωης \(Δ<0\)}
Είναι:
\begin{equation}\label{eqn:1}
    y_1(x) = e^{Ax} (\sin{Bx} + i \cos{Bx})
\end{equation}
\begin{equation}\label{eqn:2}
    y_2(x) = e^{Ax} (\sin{Bx} - i \cos{Bx})
\end{equation}

Έχουμε:
\begin{equation*}
    \begin{aligned}
        \eqref{eqn:1} + \eqref{eqn:2}           & = e^{Ax} (\sin{Bx} + i \cos{Bx})                         \\
                                                & + e^{Ax} (\sin{Bx} - i \cos{Bx})                         \\
                                                & = e^{Ax} (\sin{Bx} + i \cos{Bx} + \sin{Bx} - i \cos{Bx}) \\
                                                & = e^{Ax} (2 \sin{Bx}) \implies                           \\
        y_1(x) + y_2(x)                         & = 2 e^{Ax} \sin{Bx}                                      \\
        \frac{1}{2} y_1(x) + \frac{1}{2} y_2(x) & = e^{Ax} \sin{Bx}                                        \\
        y_3(x)                                  & = e^{Ax} \sin{Bx}
    \end{aligned}
\end{equation*}

Και:
\begin{equation*}
    \begin{aligned}
        \eqref{eqn:1} - \eqref{eqn:2}               & = e^{Ax} (\sin{Bx} + i \cos{Bx})                         \\
                                                    & - e^{Ax} (\sin{Bx} - i \cos{Bx})                         \\
                                                    & = e^{Ax} (\sin{Bx} + i \cos{Bx} - \sin{Bx} + i \cos{Bx}) \\
                                                    & = e^{Ax} (2 i \cos{Bx}) \implies                         \\
        y_1(x) + y_2(x)                             & = 2 i e^{Ax} \cos{Bx}                                    \\
        \frac{1}{2 i} y_1(x) + \frac{1}{2 i} y_2(x) & = e^{Ax} \cos{Bx}                                        \\
        y_4(x)                                      & = e^{Ax} \cos{Bx}
    \end{aligned}
\end{equation*}
\section{Παραδείγματα}
\subsection{Λύστε τις ακόλουθες Δ.Ε.}
\subsubsection{\( y''  -y' - 6y = 0 \)}
\subsubsection{\( y'' -4y' - 5y = 0 \)}
\subsubsection{\( y'' -4y' - 4y = 0 \)}
\end{document}