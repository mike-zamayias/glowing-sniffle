\documentclass[12pt, fleqn, leqno]{extreport}
\usepackage{perpage}
%\documentclass[12pt]{extreport}
\usepackage[margin=2cm,includeheadfoot,a4paper]{geometry}
\usepackage{fontspec}
%\usepackage[utf8x]{inputenc}
\usepackage[english,greek]{babel}
\usepackage{indentfirst}
\usepackage[dvipsnames]{xcolor}
\usepackage{listings}
\usepackage{titlesec}
\usepackage{amsmath, mathtools}
\usepackage{xifthen, xparse}
\usepackage{fancyhdr}
\usepackage{fancyvrb}
\usepackage[hyphens]{url}
\usepackage{hyperref}

\MakePerPage{footnote} 

%\setlength{\mathindent}{0pt}
\setlength{\headheight}{17pt}

%\renewcommand{\arraystretch}{1.5}

\titleformat{\chapter}[display]
  {\normalfont\bfseries}{}{0pt}{\Huge}

\setlength{\parskip}{0cm}
\setlength{\parindent}{1cm}

\setmainfont{[EBGaramond-Regular.ttf]}
\setmonofont{[FiraMono-Regular.otf]}

\hypersetup{
    colorlinks = true,
    linkcolor=black,
    filecolor=magenta,
    urlcolor=blue,
    pdftitle={Project 2}
}


%\definecolor{name}{model}{color-spec}

\lstdefinestyle{mystyle}{
    language=Octave,
    backgroundcolor=\color{white},   
    commentstyle=\color{teal},
    keywordstyle=\color{blue},
    numberstyle=\color{gray}\ttfamily,
    stringstyle=\color{orange},
    basicstyle=\ttfamily\footnotesize,
    breakatwhitespace=false,         
    breaklines=true,                 
    captionpos=b,                    
    keepspaces=true,                 
    numbers=left,                    
    numbersep=5pt,                  
    showspaces=false,                
    showstringspaces=false,
    showtabs=false,                  
    tabsize=2,
    frame=lines,
    framesep=0.1cm,
    rulecolor=\color{black},
    morestring=[b]"
}

\lstset{style=mystyle}

\pagenumbering{arabic}

\pagestyle{fancy}
\fancyhf{}
\fancyhead[R]{\rightmark}
\lhead{Project 2}
\chead{Αριθμητική Γραμμική Άλγεβρα}
\cfoot{\thepage}

\newcommand\rowop[1]{\scriptstyle\smash{\xrightarrow[\vphantom{#1}]{\mkern-4mu#1\mkern-4mu}}}

\DeclareDocumentCommand\converttorows
{>{\SplitList{,}}m}
{\ProcessList{#1}{\converttorow}}
\NewDocumentCommand{\converttorow}{m}
{\ifthenelse{\isempty{#1}}{}{\rowop{#1}}\\}

\DeclareDocumentCommand \rowops{m}
{\;
 \begin{matrix}
\converttorows {#1}
 \end{matrix}
 \; }


\begin{document}

\title{Αριθμητική Γραμμική Άλγεβρα\\Project 3}
\author{Μιχαήλ Ανάργυρος Ζαμάγιας\\ΤΠ5000}
\date{\today}
\maketitle
\newpage

\tableofcontents

\chapter{Άσκηση 1}

\section{Ερώτημα}
Φτιάξτε μια συνάρτηση σε Octave που να δέχεται έναν nxn πίνακα ως δεδομένα και να υπολογίζει το χαρακτηριστικό πολυώνυμο του πίνακα, τα ιδιοδιανύσματα και τις ιδιοτιμές του πίνακα, χρησιμοποιώντας τις έτοιμες εντολές του Octave. Η συνάρτηση θα πρέπει να παίρνει σαν είσοδο τον πίνακα και να δίνει σαν έξοδο ένα πίνακα με στήλες τα μοναδιαία ιδιοδιανύσματα, ένα οριζόντιο διάνυσμα με τις ιδιοτιμές στην ίδια σειρά των ιδιοδιανυσμάτων και το χαρακτηριστικό πολυώνυμο σε μορφή διανύσματος συντελεστών (αριθμητική μορφή πολυωνύμων στο Matlab και Octave). Η συνάρτηση αυτή δεν πρέπει να κάνει χρήση του συμβολικού πακέτου του Octave γιατί αυτό είναι πολύ πιο αργό από το αριθμητικό πακέτο. Δώστε τα αποτελέσματα της συνάρτησης σας στον πίνακα \eqref{eq:11}. Εδώ $\mu$ είναι το τελευταίο ψηφίο του αριθμού μητρώου σας. Επαληθεύονται οι σχέσεις του Vieta για αυτόν τον πίνακα;
\begin{equation}%eq11
    A = \begin{pmatrix}
        1 & 2 & 3   \\
        2 & 4 & 5   \\
        3 & 5 & \mu
    \end{pmatrix}\label{eq:11}
\end{equation}

\newpage
\section{Απάντηση}

Η έξοδος \lstinline[language={}]{task1.txt} του προγράμματος \lstinline[language={}]{task1.m} για τον πίνακα $ A = \begin{pmatrix}
        1 & 2 & 3 \\
        2 & 4 & 5 \\
        3 & 5 & 0
    \end{pmatrix} $:
\lstinputlisting[language={}]{task1.txt}

Η συνάρτηση \lstinline[language={}]{task1_function.m}:
\lstinputlisting{task1_function.m}

\newpage
Το πρόγραμμα \lstinline[language={}]{task1.m}:
\lstinputlisting{task1.m}


\chapter{Άσκηση 2}

\section{Ερώτημα}
Χρησιμοποιώντας την συνάρτησή σας, βρείτε τις ιδιοτιμές και τα ιδιοδιανύσματα των $A^{2}, A^{-1}, A^{2}+2A+I$. Πώς σχετίζονται οι ιδιοτιμές αυτών των πινάκων με τις ιδιοτιμές του $A$; Πώς σχετίζονται τα ιδιοδιανύσματα; Γράψτε την σχέση που συνδέει αυτούς τους πίνακες με τον διαγώνιο πίνακα $D$ που αντιστοιχεί στον $A$. Μπορείτε να χρησιμοποιήσετε αυτήν την σχέση για να ορίσετε τους πίνακες $e^A$ και $sin(A)$;


\newpage
\section{Απάντηση}

\chapter{Άσκηση 3}
\section{Ερώτημα}

Ένας πίνακας $Α$ διάστασης $n$ λέγεται διαγωνοποιήσιμος όταν υπάρχει αντιστρέψιμος πίνακας $M$ και διαγώνιος πίνακας $D$ έτσι ώστε $A = MDM^{-1}$. Όταν ο $A$ έχει $n$ ιδιοδιανύσματα, και $D$ είναι ο πίνακας με τις ιδιοτιμές στην διαγώνιο. Όταν ο πίνακας $A$ έχει $n$ διαφορετικές ιδιοτιμές, τότε έχει αναγκαστικά και $n$ ιδιοδιανύσματα και άρα αναγκαστικά διαγωνοποιείται. Όταν όμως έχει λιγότερες από $n$ ιδιοτιμές, τότε μπορεί να έχει $n$ ή λιγότερα από $n$ ιδιοδιανύσματα, και άρα μπορεί να διαγωνοποιείται ή να μην διαγωνοποιείται. Για να το δούμε με αυτό, βρείτε με το χέρι τα ιδιοδιανύσματα και τις ιδιοτιμές του πίνακα $A = \begin{pmatrix}
    1 & 1 \\
    0 & 1
\end{pmatrix}

\newpage
\section{Απάντηση}

\chapter{Άσκηση 4}
\section{Ερώτημα}

\newpage
\section{Απάντηση}

\end{document}