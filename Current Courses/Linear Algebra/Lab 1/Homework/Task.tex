\documentclass[a4paper,12pt]{article}
\usepackage{inconsolata}
\usepackage{listings}
\usepackage[margin=2cm]{geometry}
\usepackage[LGR, T1]{fontenc}
\usepackage[utf8]{inputenc}
\usepackage[greek, english]{babel}
\usepackage{alphabeta}
\usepackage{amsfonts, amsmath, amssymb}
\usepackage{fixltx2e}
\usepackage{subfig}
\usepackage{float}
\usepackage{xcolor}

\definecolor{codeblue}{RGB}{44,133,217}
\definecolor{codegray}{RGB}{138,150,150}
\definecolor{codepurple}{RGB}{0.58,0,0.82}
\definecolor{backcolour}{RGB}{255,255,255}
\lstdefinestyle{mystyle}{
    backgroundcolor=\color{backcolour},
    commentstyle=\color{codegray},
    keywordstyle=\color{codeblue},
    numberstyle=\color{codegray}\ttfamily\bfseries,
    stringstyle=\color{codepurple},
    basicstyle=\ttfamily\small\bfseries,
    breakatwhitespace=false,
    breaklines=true,
    captionpos=b,
    keepspaces=true,
    numbers=left,
    numbersep=5pt,
    showspaces=true,
    showstringspaces=false,
    showtabs=false,
    tabsize=2
}
\lstset{style=mystyle}


\begin{document}

\begin{titlepage}
    \begin{center}
        \vspace*{\fill}
        \huge{\textbf{Αριθμητική Γραμμική Άλγεβρα}}
        \vspace*{\fill}
        \vfill
        \normalsize\textbf{Ζαμάγιας Μιχάλης\\}
        \small\textbf{Εαρινό Εξάμηνο, 2020\\}
        \vfill
    \end{center}
\end{titlepage}

\tableofcontents

\newpage\section{Εργαστήρια}

\newpage\subsection{Εργαστήριο 1 - Πίνακες}
\subsubsection{Σημειώσεις}
Ένας πίνακας $A_{mxn}$, έχει \textit{\textbf{n} στήλες} και
\textit{\textbf{m} γραμμές}:
\begin{equation*}
    A_{mxn} = \begin{pmatrix}
        a_{11} & a_{12} & \cdots & a_{1n} \\
        a_{21} & a_{22} & \cdots & a_{2n} \\
        \vdots & \vdots & \ddots & \vdots \\
        a_{m1} & a_{m2} & \cdots & a_{mn}
    \end{pmatrix}
\end{equation*}
\textbf{Είδη πινάκων}:
\begin{itemize}
    \item τετραγωνικός, όπου $m = n$
    \item μη τετραγωνικός, όπου $m \neq n$
\end{itemize}
\textit{Σημείωση}: Μόνο οι τετραγωνικοί πίνακες έχουν διαγώνιους. Η κύρια διαγώνιος
είναι εκείνη όπου $i = j$, με $i \in n$ και $j \in m$. Για δευτερεύουσα την
διαγώνιο ισχύει ότι $i+j=(m*n)+1$.\\
\textbf{Μορφές τετραγωνικών πινάκων}, έστω πίνακας
$
    A_{2x2}=
    \begin{pmatrix}
        \alpha & \beta  \\
        \gamma & \delta
    \end{pmatrix}
$:
\begin{itemize}
    \item Άνω τετραγωνικός:
          $A^{'} = \begin{pmatrix}
                  \alpha & \beta  \\
                  0      & \delta
              \end{pmatrix}$
    \item Κάτω τετραγωνικός:
          $A^{'} = \begin{pmatrix}
                  \alpha & 0      \\
                  \gamma & \delta
              \end{pmatrix}$
    \item Ταυτοτικός:
          $A^{'} = I = \begin{pmatrix}
                  1 & 0 \\
                  0 & 1
              \end{pmatrix}$
    \item Μοναδιαίος:
          $A^{'} = \begin{pmatrix}
                  1 & 1 \\
                  1 & 1
              \end{pmatrix}$
    \item Μηδενικός:
          $A^{'} = \begin{pmatrix}
                  0 & 0 \\
                  0 & 0
              \end{pmatrix}$
    \item Συμμετρικός:
          $A^{'} = \begin{pmatrix}
                  \alpha & \beta  \\
                  \beta  & \delta
              \end{pmatrix}$
          ή
          $A^{'} = \begin{pmatrix}
                  \alpha & \gamma \\
                  \gamma & \delta
              \end{pmatrix}$
    \item Αντιστρέψιμος:
          $A^{-1} = \begin{pmatrix}
                  \alpha & \beta  \\
                  \gamma & \delta
              \end{pmatrix}^{-1} =
              \frac{1}{\alpha\delta-\beta\gamma}\begin{pmatrix}
                  \delta  & -\beta \\
                  -\gamma & \alpha
              \end{pmatrix}$
\end{itemize}


\newpage\subsubsection{Πρώτη Εργαστηριακή Άσκηση}
\textbf{Δίνεται ο πίνακας
$A =
    \begin{pmatrix}
        4 & 1 \\
        2 & 1
    \end{pmatrix}$
.
Βρείτε τον
$Α^{-1}$ και αποδείξτε ότι
$A*A^{-1}=Ι$,
με κώδικα και με χέρι.}

\begin{itemize}
    \item \textit{Με κώδικα}:
          \lstinputlisting[language=Octave]{task.m}
    \item \textit{Με χέρι}:\\
          Από θεωρία είναι:
          \begin{equation*}
              A^{-1} = \frac{1}{\alpha\delta-\beta\gamma}
              \begin{pmatrix}
                  \delta  & -\beta \\
                  -\gamma & \alpha
              \end{pmatrix}
          \end{equation*}
          Άρα έχουμε:
          \begin{equation*}
              \begin{split}
                  A*A^{-1}&=
                  \begin{pmatrix}
                      4 & 1 \\
                      2 & 1
                  \end{pmatrix}
                  \frac{1}{4*1-2*1}
                  \begin{pmatrix}
                      1  & -1 \\
                      -2 & 4
                  \end{pmatrix} \implies \\
                  A*A^{-1}&=
                  \frac{1}{2}
                  \begin{pmatrix}
                      4 & 1 \\
                      2 & 1
                  \end{pmatrix}\begin{pmatrix}
                      1  & -1 \\
                      -2 & 4
                  \end{pmatrix} \implies \\
                  A*A^{-1}&=
                  \frac{1}{2}
                  \begin{pmatrix}
                      (4*1+1*-2) & (4*-1+1*4) \\
                      (2*1+1*-2) & (2*-1+1*4)
                  \end{pmatrix} \implies \\
                  A*A^{-1}&=
                  \frac{1}{2}
                  \begin{pmatrix}
                      2 & 0 \\
                      0 & 2
                  \end{pmatrix} \implies \\
                  A*A^{-1}&=
                  \begin{pmatrix}
                      1 & 0 \\
                      0 & 1
                  \end{pmatrix} \implies \\
                  A*A^{-1}&=I
              \end{split}
          \end{equation*}
          \vfill
\end{itemize}




\end{document}